\subsection{The Harry Ball Theorem}
\label{sec:harry-ball}

Functions on surfaces are called vector fields.
In this section, the Gauss-Bonnet theorem is used
to related the zeros of a vector field to the topology of a surface.
We will prove that harry ball theorem which shows that 
given a harry ball, one
can not comb the ball without creating a cowlick.

\cite{rotskoff2010}


\begin{definition}[Vector Field]\label{def:vector-field}
	A vector field $v$  in an open set $U\subset S$ of a regular surface $S$
	is a correspondence which assigns to each $p\in U$ and a vector $w(p)\in T_p(S)$.
	Moreover, $v$ is differentiable at $p$ if, for some parametrization $r(u,v)$ at $p$,
	the functions $a(u,v)$ and $b(u,v)$ given by
		$$v(p)=a(u,v)r_u + b(u,v)r_v$$
	are both differentiable at $p$.
\end{definition}

Given a vector field point $p\in S$ is a \EMPH{singular point} if $v(p)=0$
and a singular point is \EMPH{isolated} if there is an open neighborhood
containing $p$ and no other singular points.



\begin{definition}[Index]\label{def:index}
Let $r$ be an orthogonal parametrization with $r(0,0)=p.$
Let $\alpha:[0,\ell]\to S$ be a closed simple regular parametrized curve with a 
simple boundary $R$.
Consider the function $\phi(t)$ with $t\in [0,\ell],$ such that records that angle 

$$2\pi I=\phi(\ell)-\phi(0)=\int_0^\ell \frac{d\phi}{dt}dt.$$
	The index $I$ of $v$.
\end{definition}

