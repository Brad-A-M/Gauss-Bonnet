

For a one dimensional curve in the plane,
how can we quantify the continuous curvature at a point on the curve?
Our definition should agree with our intuition. A straight line
should have zero curvature and the curvature of the unit circle to be one.
We would also like a way to differentiate
between turning left and turning right, that is we want curvature to be a signed quantity.

One way to define curvature is to approximate the curve at a point with a circle,
 then define the curvature to be the inverse of the radius of this circle $k=\frac{1}{r}$.
Such a circle is called the \EMPH{osculating circle}.
See \figref{osculating-circle} for an example.
For a straight line the curvature is zero, 
every point on the unit circle has curvature one 
and we can distinguish between
which way we are turning by which side of the curve the osculating circle is on,
 and as we hoped.
 
 \todo{second derivative here}

This osculating-circle view is nice because we can extend it
to  surfaces in $\R^3$, by considering the \EMPH{osculating sphere},
 the sphere that best approximates the surface at a point.
However, for saddle points on a surface it is not clear which sphere
best approximates the surface.


\todo{insert tangent space here}
Given a point on a smooth surface, we can define a normal vector at the point.
Every plane containing the normal vector will intersect the surface.
The intersection of the surface and each normal plane is a curve. 
Let $\kappa_1$ denote the maximum curvature over all normal planes
and let $\kappa_2$ denote the minimum curvature over all normal planes.
The \EMPH{Gaussian curvature} of a point on a surface is
$K=\kappa_1\kappa_2.$
This definition also satisfies the three properties that we require of a definition of
curvature.



We will consider one-dimensional curves that are the boundary of two-dimensional
surfaces in $\R^3.$ Let $\gamma$ denote such a curve parametrized by arclength.
In a surface, a \EMPH{geodesic} is a curve that is a shortest path
between two points in the surface. 
For example, on $\Sp^2$, the equator is a geodesic
and an inhabitant would view a geodesic as a straight line. 
The \EMPH{geodesic curvature} quantifies how close $\gamma$ is to being a geodesic.
Geodesics have zero geodesic curvature. 
For a formal definition see \cite{doc76}.



