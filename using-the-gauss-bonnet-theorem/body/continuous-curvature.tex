
The Gauss-Bonnet theorem relates local curvature calculations
to global topology. 
We will see several definitions of curvature.
Some definitions lend themselves to computations and others provide
more geometric intuition.
Intuitively, a straight line should have zero curvature and
 large circles should have smaller curvature than smaller circles.
 We would also like to differentiate
curving to the left and curving to the right..

In one dimension, for any point on a curve
we can approximate the curve with a circle.
The best approximating circle is the  \EMPH{osculating circle}.
 We can define the \EMPH{curvature} to be the inverse of the radius of this circle $k=\frac{1}{r}$.
See \figref{osculating-circle} for an example.
In the plane, we determine the sign by which side of the curve the osculating circle is on.
 This osculating-circle idea can be extend
to  surfaces in $\R^3$, by considering the \EMPH{osculating sphere},
But notice that at saddle points on a surface it is not clear which sphere
best approximates the surface.
The above definition provides great intuition for the curvature of curves
and surfaces but in many applications we would like a formula to compute
the curvature of a curve or surface.

A curve in $\RR^3$ is often presented as a function
$\gamma(t)=(x(t),y(t),z(t))$ that  is \EMPH{smooth} on an open interval $I$
if $\gamma'$ is continuous and $\gamma'(t)\neq (0,0,0)$ on $I$. 
If $\gamma$ is smooth it has a well-defined unit tangent vector $T(t)=\frac{\gamma'(t)}{|\gamma'(t)|}.$
A second way to define the  \EMPH{curvature} at a point is as the magnitude of the rate of change of the unit tangent vector with respect to arc length

\begin{equation} \label{eqn:kappa}
\kappa=\bigg  | \frac{T'(t)}{\gamma'(t)}\bigg |.
\end{equation}
where $t$ is arc length.

For example, take a circle of radius $r$, parameterized by $C(t)=\left(r\cos(t),r\sin(t)\right)$.
We have $\frac{dC}{dt}=C'(t)=\left(-r\sin(t),r\cos(t)\right)$ and $|C'(t)|=r$.
Then $T(t)=\left(-\sin(t),\cos(t)\right)$ and $T'(t)=\left(-\cos(t),-\sin(t)\right)$.
So, $\kappa(t)=\frac{1}{r}$ and, in this case, our definition of curvature agrees with the
osculating circle intuition given above. 
\eqnref{kappa} can be rewritten in the following more computational friendly form 
\begin{equation} \label{eqn:kappa1}
\kappa(t)=\frac{|\gamma'(t)\times \gamma''(t)|}{|\gamma'(t)|^3}.
\end{equation}

Since we traverse $\gamma$
at unit speed, $\gamma'(t)^2=1,$ and by the chain rule, $\gamma'\cdot \gamma''=0,$
so  the second derivative is orthogonal to $\gamma'$. Thus, the
vector $\gamma''=N$ is normal to the $\gamma$. 
By taking the cross product of $N$ and $T$ we obtain a vector $B$ called
the binormal vector.
The vectors $T,N$ and $B$ form the \emph{Fernet frame} of $\gamma$ a $p.$

Now we consider the curvature of a point on a surface. In one dimension we are
able parameterize our curve. Similarly, we require the ability to parameterize a surface.
A \EMPH{parameterized surface} is a map $\phi:U\subset \RR^2 \to \RR^3$ that
is differentiable, where every point in $S$ is contained in the domain of at least one map.
These maps are called \EMPH{charts}.
 The set $\phi(U)\subset \RR^3$ is called the \EMPH{trace} of $\phi$.
If the differential $d\phi_q:\RR^2\to \RR^3$ is one-to-one for all $q\in U$ then
we say $\phi$ is \EMPH{regular}. In other words, let $(u,v)$ be coordinates of $U\subset \RR^2,$
a surface is regular if $\frac{\partial\phi}{\partial u}$
and $\frac{\partial\phi}{\partial v}$ are linearly independent for all $q\in U$.


Let $S$ be a regular surface, then at each point $p\in S$ the set of tangent vectors
to parameterized curves in $S$ through $p$ forms a plane.
A \EMPH{tangent vector} to $S$ at $p$ is a map $\xi:(-\epsilon,\epsilon)\to S$ with $\xi(0)=p$.
The set of all tangent vectors is the \EMPH{tangent plane} and it corresponds to the image
of the differential map $d\phi_q(\RR^2)\subset \RR^3$ (prop. 1 \cite{doc76}).


By choosing two linearly independent paths through $p\in S$ we can obtain the explicit tangent
plane and define a normal vector $N$ at $p$.
Every plane containing the normal vector will intersect the surface.
The intersection of the surface and each normal plane is a curve in $\RR^3$
gives a one dimensional curve called the \EMPH{normal section}. 
Let $\kappa_1$ denote the maximum curvature of all normal sections 
and let $\kappa_2$ denote the minimum. 
The \EMPH{Gaussian curvature} of a point on a surface is
$K=\kappa_1\kappa_2.$



Once we choose a chart we define a clockwise orientation to be positive.
 If the clockwise
orientation can be consistently extended to the entire surface, we say
the surface is \EMPH{orientable}.
In one dimension, the curvature is the rate of change of the tangent vector.
For an orientable surface $S$, we consider the rate of change of the normal vector.
This vector is given by the map  $N:S\to \Sp^2$ that sends each
normal in $S$ to the corresponding point on $\Sp^2$ is
the \EMPH{Gauss map}.
The determinant of the derivative of the Gauss map, $dN(p)$ quantifies the rate of change of
the normal vector ($dN$ is often called the \emph{Weingarten map} \cite{Crane:2013}).
Thus, $dN_p:T_p(S)\to T_{N(p)}(\Sp^2)$, but since $T_p(S)$ and $T_{N(p)}(\Sp^2)$
are parallel we can define $dN_p$ to be a linear map on $T_p(S)$.
The determinant of $dN(p)$ is equal to \EMPH{Gaussian curvature}.



Regular surfaces are often presented as a function as $r(u,v)=(u,v,r(u,v))$
with $u,v\in(-1,1)$.
The curvature of a surface can be thought of as a the inverse of
the radius of the osculating sphere. Spheres can be expressed
as quadratic equations.  A
 \EMPH{quadratic form} to be polynomial of degree two, of the form $p(u,v)=c_1u^2+c_2uv+c_3v^2$ 
where $c_i\in R$.
We define quadratic forms using $r(u,v)$ in order to compute the curvature.


Consider a `small' parallelogram $M$ on $S$ with corners $r(u,v),r(u+\epsilon u, v), r(u,v+\epsilon v)$ 
and $r(u+\epsilon u, v+\epsilon v)$ we consider the area of this parallelogram.
A curve $u=u(t), v=v(t)$ on a regular surface, let
$s$ denote the arc length, then 
$$ds=\bigg | \frac{dr}{dt}\bigg | dt = \bigg | r_u\frac{du}{dt}+r_v\frac{dv}{dt}\bigg |dt
=\sqrt{(r_u^2 du^2+2r_ur_v du dv + r_v^2dv^2)}.$$
Let $E=r_u\cdot r_u, F=r_u\cdot r_v$ and  $G=r_v\cdot r_v$.
Then $\mathrm{I}=ds^2=Edu^2+2Fdudv +Gdv^2$ is called the \EMPH{first fundamental form}.
The rate of change of the area of $M$ is 
$dA=\sqrt{EG-F^2}dudv.$
We define an inner product on $Tp(S)$.
If $x$ and $y$ are two tangent vectors
then $\mathrm{I}(x,y)=x^T\begin{bmatrix}
E & F \\
F & G 
\end{bmatrix}y.$






\todo{Do we need this? Let $S_1$ and $S_2$ be two surfaces with $\sigma:V\subset S_1\to S_2$ a differentiable map.
At $p\in S_1$ the map $d\sigma_p:T_p(S_1)\to T_{\sigma(p)}(S_2)$ is called the
\EMPH{differential} of $\sigma$ at $p$.}

The unit normal vector $n=\frac{r_u\times r_v}{|r_u\times r_v|}$.
Then let $L=r_{uu}\cdot n, M=r_{uv}\cdot n$ and $N=r_{vv}\cdot n$ the
\EMPH{second fundamental form} is $\mathrm{I\!I}=Ldu^2+2Mdudv+Ndv^2$.
Another inner product is given by $\mathrm{I\!I}(x,y)=x^T\begin{bmatrix}
L & M \\
M & N 
\end{bmatrix}y.$
Then the Gaussian curvature of a surface is also given by
$K=\frac{\det(\mathrm{I\!I})}{\det(\mathrm{I})}.$


We now give a useful formula for computing curvature.

\begin{theorem}[Gaussian Curvature]\label{thm:log-curve}
	Let $f:U\in\R^2\to S$ be an isothermal parameterization of a regular surface.
	The Gaussian curvature, $K,$ of $S$ is 
		\begin{equation}\label{eqn:log-curve}
			K=\frac{-1}{2\lambda}\delta(\log \lambda)
		\end{equation}
\end{theorem}
\begin{proof}
\end{proof}

\subsubsection{Geodesics Curvature}

Shortest paths play an important role in many computational problems.
In a surface, a \EMPH{geodesic} is a curve that is a shortest path
between two points in the surface. 
For example, on $\Sp^2$ great circles are geodesic.
We would like to define the curvature as it would be seen from
someone living on a surface.

Let $U$ be a parameterized chart on a surface $S$ with vector $n(u,v)$ normal
to the surface
and let $\gamma(t)$ be a curve in $U$, with Frenet frame $T,N,B$.
Then $V=n(\gamma(t)\times T$ is in the tangent plane of the surface since
it is perpendicular to $n$. We now have a new orthonormal basis at a point on $\gamma$
namely, $T,V,n$. 
We would like to measure how the rate of change of the tangent vector $T$ with respect to $V$.
The \EMPH{geodesic curvature} is  
\begin{equation} \label{eqn:geodesic}
	k_g=\langle \gamma''(t),V(t)\rangle
\end{equation}
For an alternative equivalent definition see \cite{doc76}.
\todo{non great circle on sphere image}



