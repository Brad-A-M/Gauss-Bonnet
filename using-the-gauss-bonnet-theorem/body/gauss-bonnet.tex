\subsection{Proof}
\label{sec:proof}


We present a proof of the Gauss-Bonnet theorem due to \cite{upadhyay2015}.



\begin{theorem}[Discrete surfaces without boundary]\label{thm:g-b-discete-bdy}
For a combinatorial surface $S$ with no boundary

$$\sum_{v\in V} d(v)=2\pi \chi(S)$$
where $d(v)$ is the angle defect.
\end{theorem}

\begin{proof}

For each vertex $v$ in $S$,
let $\beta_1,\beta_2,\ldots,\beta_n$ denote the interior angles
containing $v$.
The curvature at vertex equals $(2-n)\pi +\sum_{i=1}^n \beta_i$.
\todo{explain}

The defect  at a vertex is equal to the curvature so,
 $$d(v)=(2-n)\pi +\sum_{i=1}^n \beta_i.$$

Let $n(v)$ denote the number of interior angles incident to $v$, the summing over all vertices gives
$$\sum_{v\in V} d(v)=\sum_{v\in V}2\pi - \sum_{v\in V}n(v)\pi+\sum_{v\in V}\sum_{i=1}^n \beta_i.$$

The first term on the right hand side is $2\pi V$. Each edge is incident with two vertices, so the second
term is $2\pi E$. So far so good!
We rearrange the angles in the third term,
$$ \sum_{v\in V}\sum_{i=1}^n \beta_i=\sum_F\sum_{v\in F}(\pi-\alpha_f).$$
Since we have a combinatorial surface, each face is a triangle,
so $\sum_{v\in F}(\pi-\alpha_f)=3\pi-\pi=2\pi.$
Thus, $$\sum_{v\in V} d(v)=2\pi V-2\pi E+2\pi F=2\pi \chi(S)$$ as desired.
\end{proof}

The above proof can be extended to the case where $S$ has a boundary
by gluing a copy of $S$ to itself along the boundary.

\begin{theorem}[Discrete surfaces without boundary]\label{thm:g-b-discete}
For a combinatorial surface $S$ with no boundary

$$\sum_{v\in V_{\text{int}}} d(v)+\sum_{v\in\partial S}k_g(v)=2\pi \chi(S)$$
where $k_g(v)$ is the exterior angle at $v$ 
$k_g(v)=\pi-\sum_i\alpha_i$ for $\alpha_i$ incident to $v$.
\end{theorem}

\begin{proof}
Take a copy of $S$ and attach it to itself along the boundary.
This creates the surface $2S$ without boundary. Notice,
when we copy $S$ we create two copies of the boundary, and when
we glue we remove one copy of the boundary.
Thus, $\chi(2S)=2\chi(S)-\chi(\partial S).$
Since, $\partial S$ is piecewise linear the number of vertices and
edges are equal and there are no faces, so $\chi(\partial S)=0$
and 

\begin{equation} \label{eqn:glue}
\chi(2S)=2\chi(S).
\end{equation}

For $v$ a vertex on the boundary, by definition $k_g(v)$ is half
the discrete gaussian curvature of $v$ in $2S.$

Thus,

$$\sum_{v\in 2S}d(v)=2\left(\sum_{v\in S_{\text{int}}}d(v)+\sum_{v\in \partial S} k_g(v)\right) =2\pi  \chi(2S).$$
Applying \eqnref{glue},

$$\sum_{v\in S}d(v)+\sum_{v\in \partial S} k_g(v)=2\pi  \chi(S)$$
as desired.

\end{proof}