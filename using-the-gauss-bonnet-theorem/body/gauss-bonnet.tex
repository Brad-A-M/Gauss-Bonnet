\section{Proof of Gauss-Bonnet}
\label{sec:proof}


In this section, we prove the theorem.

\subsection{Discrete 1}
Nice discrete proof from \cite{upadhyay2015} and \cite{Crane:2013}.

\begin{theorem}[Discrete surfaces without boundary]\label{thm:g-b-simple}
For a combinatorial surface $S$ with no boundary

$$\sum_{v\in V} d(v)=2\pi \chi(S)$$
where $d(v)$ is the angle defect.
\end{theorem}

\begin{proof}
For each vertex $v$ in $S$,
let $\beta_1,\beta_2,\ldots,\beta_n$ denote the interior angles
containing $v$.
The angle defect equals the curvature which is also
equal to the difference between $\sum_i \beta_i $ and $(n-2)\pi$. 
So, $$d(v)=(2-n)\pi +\sum_{i=1}^n \beta_i.$$

Let $n(v)$ denote the number of interior angles incident to $v$, the summing over all vertices gives
$$\sum_{v\in V} d(v)=\sum_{v\in V}2\pi - \sum_{v\in V}n(v)\pi+\sum_{v\in V}\sum_{i=1}^n \beta_i.$$

The first term on the right hand side is $2\pi V$. Each edge is incident with two vertices, so the second
term is $2\pi E$. So far so good!
We rearrange the angles in the third term,
$$ \sum_{v\in V}\sum_{i=1}^n \beta_i=\sum_F\sum_{v\in F}(\pi-\alpha_f).$$
Since we have a combinatorial surface, each face is a triangle,
so $\sum_{v\in F}(\pi-\alpha_f)=3\pi-\pi=2\pi.$
Thus, $\sum_{v\in V} d(v)=2\pi v-2\pi E+2\pi F=2\pi \chi(S)$ as desired.
\end{proof}