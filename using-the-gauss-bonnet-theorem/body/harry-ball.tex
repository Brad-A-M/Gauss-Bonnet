\subsection{The Harry Ball Theorem}
\label{sec:harry-ball}

Functions on surfaces are called vector fields.
In this section, the Gauss-Bonnet theorem is used
to relate the zeros of a vector field to the topology of a surface.
We will prove that harry ball theorem which shows that 
given a harry ball, one
can not comb the ball without creating a cowlick.




\begin{definition}[Vector Field]\label{def:vector-field}
	A vector field $v$  on an open set $U\subset S$ of a regular surface $S$
	is a correspondence which assigns a vector $w(p)\in T_p(S)$ to each $p\in U$.
	Moreover, $v$ is differentiable at $p$ if, for some parametrization $r(u,v)$ at $p$,
	the functions $a(u,v)$ and $b(u,v)$ given by
		$$v(p)=a(u,v)r_u + b(u,v)r_v$$
	are both differentiable at $p$.
\end{definition}

A point $p\in S$ in vector field is a \EMPH{singular point} if $v(p)=0$
moreover, a singular point is \EMPH{isolated} if there is an open neighborhood
containing $p$ and no other singular points. These, and the following definition,
extend to higher dimensions.

Let $f:\Sp^n\to \Sp^n$ be a continuous map. Let $H_n(\cdot)$ dentote  
the $n$th  homology group, then $f$ induces a homomorphism
$f_*:H_n(\Sp^n)\to  H_n(\Sp^n)$ and  $H_n(\Sp^n)\cong \ZZ$.
 Every homomorphism from $\ZZ$ to itself is of the form $f_*:n\mapsto kn$ for some 
$k\in \ZZ$ the integer $k$ is the \EMPH{degree} of the map $f$.

Isolated singular points give a map from the circle to itself.
Let $\sigma$ be a circle surrounding a singular point $p$.
Since $v(p)$ is isolated we can choose the radius of $\sigma$ to be 
small enough so that $v|_\sigma\neq 0$. 
The \EMPH{index} of a isolated critical point $p$ of $v$ is the degree
of the map $f:\sigma\to \Sp^1$ where$f(p)=v(p)/||v(p)||$.





\begin{theorem}[Poincar\'{e}-Hopf Theorem]\label{thm:poincare-hopf}
	Let $S$ be a closed and bounded regular surface and let $v$ be a vector field
	on $S$ such that every zero of $v$ is isolated. If $S$ has a boundary, 
	then assume $v$ is outward normal on the boundary. Then
	$$\sum_iI(v_i)=\chi(S)$$
	where $v_i$ denotes the set of isolated zeros.
	
\end{theorem}

We now prove that given a ball with a hair attached to each point, you can not
comb the ball without creating a cowlick.

\begin{theorem}[The Harry Ball Theorem]\label{thm:harry-ball}
	Let $v$ be a vector field on the sphere $\Sp^2$.
	Then there is a $p\in \Sp^2$ such that $v(p)=\vec{0}$.
\end{theorem}

\begin{proof}
	For a contradiction, assume that $v(p)\neq 0$ for all  $p\in \Sp^2$,
	this would give a vector field on $\Sp^2$ with no singular points.
	However, the Euler characteristic of $\Sp^2$ is two and
	\thmref{poincare-hopf} implies that any vector field on $\Sp^2$ has at
	least two isolated singular points.
	
\end{proof} 
\cite{rotskoff2010}