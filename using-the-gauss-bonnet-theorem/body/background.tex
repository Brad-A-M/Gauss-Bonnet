%!TEX TS-program = pdflatex
%!TEX encoding = UTF-8 Unicode
%!TeX spellcheck = en-US
%!TeX root = ..\stacs22.tex

%----------------------- Background -------------------------------------
\section{Background}
\label{sec:background}


In this section, we introduce concepts and definitions that are used throughout the paper.
We assume the readers are familiar with the basic terminology for curves and surfaces.
% Additional definitions are included in \appendref{appendix}.
% \note{Like what?}

\subsection{Curves}

A \EMPH{closed curve} in the plane is a continuous map 
$\EMPH{$\gamma$}: \mathbb{S}^1 \to \R^2$, and a \EMPH{path} in the plane is a 
continuous map $\EMPH{$\zeta$}: [0,1]\to \R^2$.
A path $\zeta$ is \emph{closed} when $\zeta(0)=\zeta(1)$.
We will consider the surface obtained by removing a finite number
of points from the plane. The removed points are \EMPH{punctures}
and paths with end points on distinct punctures are \EMPH{arcs}.
%
In this work, we are presented with a \EMPH{generic} curve,
where there are a finite number of self-intersections, each of
which is transverse and no three strands cross at the same point.
The image of a generic closed curve is naturally associated with a 4-regular plane graph;
we abuse notation and denote the image graph as {$\curve$} as well.
The self-intersection points of a curve are \EMPH{vertices},
the paths between vertices are \EMPH{edges}, and the connected
components of the complement of the curve are \EMPH{faces}.

%
In this paper, we choose an arbitrary starting point $\gamma(0)=\gamma(1)$ and 
orientation for $\curve$.
We define the \EMPH{dual graph $\curve^*$} by choosing
an arbitrary point $f^*$ in each face, then for every edge $e$
choose a path $e^*$ between the two points in the faces incident to $e$
such that $e^*$ intersects $e$ once transversely and $e^*$ does not intersect
any other edges of $\curve$. The resulting graph is another plane graph.




\paragraph*{Tree-Cotree Decomposition.}
Let $T$ be a spanning tree of $\curve$, and let $\curve^*$ be the dual of $\curve$.
The tree $T$ partitions the edges into two subsets, $T$ and $T^* \coloneqq E \setminus T$.
The edges in $T^*$ define a spanning tree of $\curve^*$ called the \EMPH{cotree}.
The partition of the edges by $T \sqcup T^*$ is called the \EMPH{tree-cotree decomposition}.

We call a rooted spanning cotree $T^*$ of $\curve^*$ a \EMPH{breadth-first search} tree 
(abbreviated as BFS-tree) if it can be generated from a breadth-first search rooted at the 
vertex in $\curve^*$ corresponding to the unbounded face in $\curve$.
Each face $f$ of~$\curve$ is a vertex in a breadth-first search tree $T^*$, we associate $f$ 
with the unique edge incident to $f^*$ in the direction of the root.
Thus, there is a correspondence between edges of $T^*$ and faces of $\curve$.


\subsection{Homotopy and Isotopy}

A \EMPH{homotopy} between two closed curves $\gamma_1$ and $\gamma_2$ that 
share a point $p_0$ is a continuous map $H\colon [0,1]\times \Sp^1 \to \mathbb{R}^2$ 
such that $H(0,\cdot)=\gamma_1$, $H(1,\cdot)=\gamma_2$, and $H(s,0)=p_0=H(s,1)$.
%\footnote{This is known as \EMPH{free homotopy} in standard topology textbook.}
We can define homotopy between two paths similarly, where the two endpoints are fixed
 throughout the continuous morph.
Notice that homotopy between two closed curves as \emph{closed curves} and the 
homotopy between them as \emph{closed paths} with an identical starting points are different.
\brad{why make the above distinction?}
We restrict ourselves to $\mathbb{R}^2$ where any two closed curves are homotopic.
%
A \EMPH{isotopy} between two injective paths $\zeta_1$ and $\zeta_2$ in $\R^2$ is
a homotopy $H\colon [0,1]\times [0,1] \to \R^2$ with $H(0,\cdot)=\zeta_1,
 H(1,\cdot)=\zeta_2$ such that $H(t,\cdot)$ is injective for all $t\in[0,1]$.
The notion of isotopy naturally extends to a collection of paths.

We can think of $\curve$ as a topological space and consider the \EMPH{fundamental 
group $\pi_1(\curve)$} of $\curve$.
Elements of the fundamental group are called \EMPH{words}, consisting of letters that 
correspond to equivalence classes of homotopic closed paths in $\curve$.
%By proposition 1A.2 in \cite{hatcher},   % I don't think this needs citation
The fundamental group of $\curve$ is a free group with basis consisting of the classes 
corresponding to the cotree edges of any tree-cotree decomposition of $\curve$.




 Let $H$ be a homotopy between curves $\gamma_1$ and $\gamma_2$.
 Let $\EMPH{$\#H^{-1}(x)$} \colon \R^2 \to \Z$ be the function that assigns to each $x\in\R^2$ 
 the number of times the intermediate curves $H(s)$ sweep over $x$.
 The homotopy area of $H$ is
 \[
     \EMPH{$\Area(H)$} \coloneqq \int_{\R^2} \#H^{-1}(x) \, d\!x.
 \]

 The minimum area homotopy between $\gamma_1$ and $\gamma_2$ is the infimum of
  the homotopy area over all homotopies.
 We denote this by
 \[
     \EMPH{$\Area_H(\gamma_1,\gamma_2)$} \coloneqq \inf_{H} \, \Area(H).
 \]
 When $\gamma_2$ is the constant curve at a specific point $p_0$ on $\gamma_1$, 
 define $\EMPH{$\Area_H(\gamma)$} \coloneqq \Area_H(\gamma,p_0)$.
 See \figref{eight} for an example of a homotopy.


 \begin{figure}[htb]
         \centering
        \begin{subfigure}[b]{0.15\textwidth}
         \includegraphics[width=\textwidth]{homotopy/good-and-bad}
         \caption{A curve.}
 	 \label{fig:homotopy}
       \end{subfigure}
         \hspace{1cm}
         \begin{subfigure}[b]{0.15\textwidth}
         \includegraphics[width=\textwidth]{homotopy/good-and-bad-homotopy}
         \caption{Sweep $f_3$.}
         \end{subfigure}
          \hspace{1cm}
            \begin{subfigure}[b]{0.15\textwidth}
        \includegraphics[width=\textwidth]{homotopy/eight-homotopy-1}
         \caption{Sweep $f_3$.}
 	 \label{fig:continue}
         \end{subfigure}
         \hspace{1cm}
         \begin{subfigure}[b]{0.15\textwidth}
        \includegraphics[width=\textwidth]{homotopy/edge-un-spike}
         \caption{Avoid $f_2$.}
          \label{fig:un-spiked}
        \end{subfigure}
		\caption{(a) A generic closed curve in the plane.
 		(b) We see a homotopy that sweeps over the face $f_3$.
 		(c) The homotopy continues contracting sweeping of $f_3$ again.
 		(d) The curve now encloses a small area.
		The homotopy avoids sweeping over the face $f_2$.
 		This is a minimum area homotopy for the curve.
 		The minimum area is $\Area(f_1) + 2\cdot\Area(f_3)$.
 		\label{fig:eight}}
 \end{figure}


 For each $x\in \RR \setminus \curve$, the \EMPH{winding number} of $\gamma$ at $x$, 
 denoted as \EMPH{$\wind(x,\gamma)$}, is the number of times $\gamma$ ``wraps around'' $x$,
  with a \emph{positive} sign if it is counterclockwise, and \emph{negative} sign otherwise.
Put it differently, the winding number is the number of times $\gamma$ passes through an 
arbitrary ray from $x$ to infinity from right to left, minus the number of times $\gamma$ 
passes the ray from left to right.
%A pleasant description is given in~\cite{chinn-steenrod}.
The winding number is a constant on each face.
 Define the \EMPH{winding area} of $\gamma$ as the integral
 \[
 \EMPH{$\Area_W(\gamma)$} \coloneqq \int_{\R^2} \abs{\wind(x,\gamma)} \, d\!x
 = \sum_{\text{face $f$}} \abs{\wind(f,\gamma)} \cdot \Area(f).
 \]


% Next we describe a decomposition of a curve into simple cycles
% based on the depth that is due to Chang-Erickson~\cite{changErickson17}.
% Let $d$ denote the depth of $\gamma$, for all $1<j<d$, let $R_j$ denote
% the set of points with $\depth(x,\gamma)\geq d+1-j$.
% Consider a open neighborhood of the closure of $R_j\cup \tilde{R}_{j-1},$
 %call this $\tilde{R}_j$. Each $\tilde{R}_j$ is the disjoint union of closed disks.
 %Then $\tilde{R}_0$ is the empty set,
% $\tilde{R}_1$ is disk containing all points of depth $d,
% \tilde{R}_2$ is disk containing all points of depth $d-1$ and so on.
%Finally, $\tilde{R}_d$ is a disk containing the entire curve.
 %We call such a decomposition the \EMPH{depth cycles}
 %of a curve $\gamma$.


The \EMPH{depth} of a face $f$ is the minimal number of edges crossed 
 by a path from $f$ to the exterior face.
 The depth is a constant on each face.
 We say the depth of a curve is equal the maximum depth over all faces.
 We define the \EMPH{depth area} to be
\[
 \EMPH{$\Area_D(\gamma)$} \coloneqq \int_{\R^2} \depth(x,\gamma)\, d\!x
 = \sum_{\text{face $f$}} \depth(f) \cdot \Area(f).
 \]

 Chambers and Wang~\cite{cw2013} showed that the winding area gives a lower 
 bound for the minimum homotopy area.
There exists a homotopy with area $\Area_D(\gamma)$, 
one such homotopy can be constructed by smoothing the curve
at each vertex, then contracting each simple cycle.
 Therefore we have
 \[
     \Area_W(\gamma) \leq \Area_H(\gamma) \leq \Area_D(\gamma).
 \]


 \subsection{Self-Overlapping Curves}

 A generic curve $\gamma$ is \EMPH{self-overlapping} if there is an immersion of the disk
 $F:\mathbb{D}^2\rightarrow \RR^2$ such that $\curve= F|_{\partial\!D^2}$.
 We say a map $F$ \EMPH{extends} $\gamma$.
The image $F(\mathbb{D}^2)$ is the \EMPH{interior} of $\gamma$.
 Intuitively, the image the disk $\mathbb{D}^2$ has two sides, one colored with 
 blue and the other with red.
 A curve $\gamma$ is self-overlapping if it is the boundary of a disk $\mathbb{D}^2$ 
 with only one color facing up~\cite{mukherjee2014}.
 There are several other ways to define self-overlapping curves
  \cite{evansFasyWenk,titus,shor-van-wyk,so-graphics}.
  Self-overlapping curves have the property that the minimum 
  homotopy area equals the winding area:
 $\Area_W(\gamma) = \Area_H(\gamma)$~\cite{fkw2017}.

 
Polynomial algorithms for determining if a curve is self-overlapping
are given in  \cite{blank, shor-van-wyk}. These works are not only interested
 in determining if a curve is self-overlapping,
 they also determine the number of equivalent extensions.
 See \figref{extensions} for an example of a curve with two extensions
 that are not diffeomorphic.
 In related work, Eppstein and Mumford show it is NP-complete to determine
 if a curve is the boundary of an immersed surface with boundary in $\R^3$
  \cite{eppsteinMumford}.



 For any curve, the \EMPH{intersection sequence}%
 \footnote{also known as the unsigned Gauss code~\cite{changErickson17, gauss}}
 $[\gamma]_V$ is a cyclic sequence of vertices $[v_0, v_1, \ldots , v_{n-1}]$ 
 with $v_n = v_0$, where each $v_i$ is an intersection point of $\gamma$.
 Each vertex appears exactly twice in $\gamma_V$.
 Two vertices $x$ and $y$ are \EMPH{linked} if the two appearances of $x$ and 
 $y$ in $\gamma_V$ alternate in cyclic order:  $\dots x \dots y \dots x \dots y \dots \,$.
 We use the standard sign convention~\cite{changErickson17, gauss} that 
 $\textrm{sgn}(v_i) = +1$ if the first traversal through $v_i$ crosses the second 
 traversal from right to left, and $\textrm{sgn}(v_i)=-1$ otherwise.



 A pair of symbols of the same vertex $x$ but opposite signs induces two 
 natural subcurves generated by \EMPH{smoothing} the vertex $x$; 
 see \figref{vertex-cut} for an example.
 (In this work, every smoothing is done in the way that respects the orientation 
 and splits the curve into two subcurves.)
 A \EMPH{vertex pairing} is a collection of pairwise unlinked vertex pairs in $[\gamma]_V$.

 A \EMPH{self-overlapping decomposition} $\Gamma$ of $\gamma$
 is a set of paired vertices such that
 the induced subcurves are self-overlapping;
 see \figref{total-decomp} and \figref{intersections-p1} for examples.
 The subcurves that result from a vertex pairing are not necessary self-overlapping;
 see \figref{intersections-p0} for an example.
 For $\Gamma$ a self-overlapping decomposition of $\gamma$,
 denote the set of induced subcurves by $\{\gamma_i\}_{i=1}^\ell$.
 Since each $\gamma_i$ is self-overlapping the minimum homotopy
 area is equal to the winding area.
 We define the \EMPH{area of of self-overlapping decomposition}
 to be

 \[
 	\EMPH{$\Area_\Gamma(\gamma)$}:=\sum_{i=1}^\ell \Area_W(\gamma_i).
 \]




 \begin{figure}[htb]
  \centering
      \begin{subfigure}[b]{0.22\textwidth}
      \includegraphics[width=\textwidth]{smoothing/mouse-base}
      \caption{A closed curve $\curve$.}
     \label{fig:intersections}
      \end{subfigure}
      \begin{subfigure}[b]{0.22\textwidth}
      \includegraphics[width=\textwidth]{smoothing/mouse-all-paired}
      \caption{All pairings.}
      \label{fig:total-decomp}
      \end{subfigure}
     \begin{subfigure}[b]{0.22\textwidth}
      \includegraphics[width=\textwidth]{smoothing/mouse-not-so}
      \caption{Pairing at $v_0$.}
      \label{fig:intersections-p0}
      \end{subfigure}
      \begin{subfigure}[b]{0.22\textwidth}
      \includegraphics[width=\textwidth]{smoothing/mouse-so2}
      \caption{Pairing at $v_1$.}
      \label{fig:intersections-p1}
      \end{subfigure}

  \caption{
     (a) A curve $\gamma$ with labeled intersection sequence
      $\gamma_V=[v_0,v_1,v_1,v_2,v_2,v_0]$.
      (b) All vertices are paired giving a smoothing at every vertex in the intersection sequence.
      (c) A vertex pairing such that one of the subcurves is not self-overlapping.
      (d) A vertex pairing such that both subcurves are self-overlapping.}
  \label{fig:vertex-cut}
 \end{figure}



 Fasy, Karakoç, and Wenk~\cite{fkw2017} proved the following structural theorem:
 \begin{theorem}[Self-Overlapping Decomposition~{\cite[Theorem 20]{fkw2017}}]
 \label{thm:structual}
 There exists a self-overlapping decomposition such that the area of the self-overlapping
 decomposition has the minimum area over all homotopies.
 \end{theorem}
 The above theorem implies if one finds the minimum area over all self-overlapping decompositions,
 then one has found the minimum homotopy area.



