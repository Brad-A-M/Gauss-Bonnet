\section{Classifying Surfaces}
\label{sec:classiify}

This is dumb, we aren't using the theorem
just the Euler characteristic.


Given two surfaces one often wants to know
if there exists a homeomorphism between them.
\todo{find specific application? Maybe texture mappings}
By the classification of surfaces the topology of
an oriented surface is determined by is Euler characteristic.
Thus, we can determine if a homeomorphism exists simply by
counting the vertices, edges and faces of each surface.

Surfaces are often stored in a half-edge data structure \cite{Marks,Crane:2013}.
For a surface, the half-edge consists of a list for the vertices, edges and faces.
The vertex list includes the coordinates of $v$ and a pointer to an incident half-edge.
The face list includes a pointer to an incident half-edge on the outer boundary.
The edge list is more involved, 
For each edge we record pointers to the origin, an incident face and 
an edge with the opposite orientation. This edge is the \EMPH{twin} edge of a given edge.
Thus, each edge in the triangulation is stored in two half-edges, hence the name
of the data structure.

Consider the surfaces given as half-edge data structures in \tabref{torus} and \tabref{sphere},
are these surfaces homeomorphic?

\begin{center}
\begin{table}
	\caption{\label{tab:torus}A surface is given in a half-edge data structure}
	\begin{tabular}{|c|c|c|}\label{tab:torus}
		
		hy & ha& b\\
		\hline
	\end{tabular}
\end{table}
\end{center}