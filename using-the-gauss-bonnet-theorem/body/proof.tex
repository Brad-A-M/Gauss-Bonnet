\subsection{Proof}
\label{sec:proof}


We present a proof of the Gauss-Bonnet theorem due to Upadhyay \cite{upadhyay2015}.



\begin{theorem}[Discrete surfaces without boundary]\label{thm:g-b-discete-bdy}
For a combinatorial surface $S$ with no boundary

$$\sum_{v\in V} K(v)=2\pi \chi(S)$$
where $K(v)$ is the discrete curvature at $v$.
\end{theorem}

\begin{proof}

For each vertex $v$ in $S$,
let $\beta_1,\beta_2,\ldots,\beta_{\deg{(v)}}$ denote the interior angles
containing $v$.
The curvature at vertex equals $(2-\deg{(v)})\pi +\sum_{i=1}^{\deg{(v)}} \beta_i$.

The curvature at a vertex is ,
 $$K(v)=(2-\deg{(v)})\pi +\sum_{i=1}^{\deg{(v)}} \beta_i.$$

Summing over all vertices in $S$ gives
$$\sum_{V} K(v)=\sum_{v\in V}2\pi - \sum_{v\in V}\deg{(v)}\pi+\sum_{v\in V}\sum_{i=1}^{\deg{(v)}} \beta_i.$$

The first term on the right hand side is $2\pi |V|$. Each edge is incident with two vertices, so the second term is $2\pi |E|$. 
In the third term, we use \eqnref{switcheroo} to rewrite $\beta_i$ as $\pi-\alpha_i$.

$$ \sum_{V}\sum_{i=1}^{\deg{(v)}} \beta_i= \sum_{v\in V}\sum_{i=1}^{\deg{(v)}} (\pi-\alpha_i).$$
We are summing $\pi$ minus each angle incident to each vertex, 
instead of summing around each vertex we can sum around the angle in each face.
Each angle in $S$ is still being counted exactly once. This gives
$$\sum_{V}\sum_{i=1}^{\deg{(v)}} (\pi-\alpha_i)=\sum_F\sum_{i=1}^3(\pi-\alpha_i).$$

Since we have a combinatorial surface, each face is a triangle,
so $\sum_{i=1}^3(\pi-\alpha_i)=3\pi-\pi=2\pi.$
Thus, $$\sum_{v\in V} K(v)=2\pi |V|-2\pi |E|+2\pi |F|=2\pi \chi(S)$$ as desired.
\end{proof}

The above proof can be extended to the case where $S$ has a boundary
by gluing a copy of $S$ to itself along the boundary.

\begin{theorem}[Discrete surfaces without boundary]\label{thm:g-b-discete}
For a combinatorial surface $S$ with no boundary

$$\sum_{v\in S_{\text{int}}} K(v)+\sum_{v\in\partial S}k_g(v)=2\pi \chi(S)$$
where $k_g(v)$ is the exterior angle at $v$ 
$k_g(v)=\pi-\sum_i\alpha_i$ for $\alpha_i$ incident to $v$.
\end{theorem}

\begin{proof}
Take a copy of $S$ and attach it to itself along the boundary.
This creates the surface $2S$ without boundary. Notice,
when we copy $S$ we create two copies of the boundary, and when
we glue we remove one copy of the boundary.
Thus, $\chi(2S)=2\chi(S)-\chi(\partial S).$
Since, $\partial S$ is piecewise linear the number of vertices and
edges are equal and there are no faces, so $\chi(\partial S)=0$
and 

\begin{equation} \label{eqn:glue}
\chi(2S)=2\chi(S).
\end{equation}

For $v$ a vertex on the boundary, $k_g(v)$ is half
the discrete gaussian curvature of $v$ in $2S.$

Thus,

$$\sum_{v\in 2S}K(v)=2\left(\sum_{v\in S_{\text{int}}}K(v)+\sum_{v\in \partial S} k_g(v)\right) =2\pi  \chi(2S).$$
Applying \eqnref{glue},

$$\sum_{v\in S}K(v)+\sum_{v\in \partial S} k_g(v)=2\pi  \chi(S)$$
as desired.

\end{proof}