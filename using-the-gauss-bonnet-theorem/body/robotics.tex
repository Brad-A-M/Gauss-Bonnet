\subsection{Robotics}
\label{sec:robotics}
In may 2023 I asked chatgbt to give me some applications of the Gauss-Bonnet theorem and returned
that there were applications to robotics. I asked for some references it gave me references
that were made up. But the suggestion lead me to the following
not made up application of the Gauss-Bonnet theorem in robotic 
route planning is given by K.-L. Wu et. al. in \cite{wu_path_2016}.
Suppose have a robot navigating a 3D\todo{2 or 3 manifold?} terrain with a single obstacle
and we wish to plan trips for our robot.
In this application, a terrain is a smooth manifold \todo{not defined} with tangent planes
at every point. An obstacle is modeled by a hazardous ball with a grade depending on the radius.
Assume that checking if a path intersects the obstacle can be done in constant time.


Overview of their procedure.
We are given two points $s$ and $t$ on a ?-manifold $M$.
First, compute a geodesic path from  $s$ to $t$ call this path $\gamma(x)$.
If the $\gamma$ does not intersect the obstacle we are done.
Otherwise, the $\gamma$ intersects the obstacle
we call the initial intersection point between our path
and the boundary of the obstacle $p$.
Construct the tanged plant $TpM$ at $p$.
Choose a vector $v\in TpM$ and a value $\alpha$ for the magnitude of $v$.
Next, define two points $\alpha_{\ell}$ and $\alpha_{u}$
to be in the directions of $v$ and $-v$ at a distance of $\alpha$
from $p$ in the tangent plane. Then project the points   $\alpha_{\ell}$ and $\alpha_{u}$
onto the surface? to obtain the points $q$ and $r$.
We next compute four new  geodesics $g_1(s)$ from $s$ to $q$,
$g_2(s)$ from $q$ to $t$, 
$f_1(s)$ from $s$ to $r$ and 
$f_2(s)$ from $r$ to $t$. Let $\gamma_g=g_2\circ g_1$ and $\gamma_f=f_2\circ f_1$.

We then use the Gauss-Bonnet theorem to decide which alternative path is best.
If the edges of a triangle are all geodesic then we have a \EMPH{geodesic triangle} $\tau\subset M$.
We have two geodesic triangles, $\gamma, g_1,g_2$ and $\gamma,f_1,f_2.$

Here we have cusps at the intersection points of the geodesics,
to account for this, our Gauss-Bonnet is 
\begin{equation}\label{eqn:b-g-angles}
\int \int_{\tau} K dA +\sum_{i=1}^3(\pi-\theta_i)+\sum_{i=1}^3 \int k_gds =2\pi
\end{equation}
where $\theta_i$ are the interior angles of $\tau$.
Since we are on geodesics $\int kgds =0$ and we can rearrange
\eqnref{b-g-angles} to obtain

\begin{equation} \label{eqn:interior-angles}
\theta_1+\theta_2+\theta_3 = \pi +\int \int_{\tau} K dA.
\end{equation}

We can estimate the curvature based on the sum of the angles of
$\sum_{i=1}3\theta$.
They show that f $K=0$ on $\tau$ then $\gamma_f$ and $\gamma_g$
are identical and there exists an $\alpha^*$ that makes them shortest.
If $\sum_{i=1}^3\theta_i>\pi$ then is $\int_?K>0$ on all of $\tau$ then and if $\sum_{i=1}^3\theta_i<\pi$
then $K<0$ on (average) all of $\tau$.
The authors state that we should avoid negative curvature because it
is ``energy-consuming ascending and descending motions required"
and would require the robot have ``better mobility and maneuverability".
\todo{I don't see why}.


