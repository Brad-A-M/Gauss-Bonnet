\subsection{Triangulations}

Here we define simplicial complexes and triangulations.


\begin{definition}[Topological Space \cite{munkres}]
A \EMPH{topology} is a pair $(X,\tau)$, where $X$ is a set and
 $\tau$ is a collection of subsets $X$
satisfying:
	\begin{itemize}
		\item $\emptyset$ and $X$ are in $\tau.$
		\item the union of \emph{any} subcollection of elements in $\tau$ is  in $\tau.$
		\item the intersection of any \emph{finite} subcollection of elements in the $\tau$ is in $\tau.$
	\end{itemize}
A set $X$ with a specified topology $\tau$ is called a \EMPH{topological space}.
\end{definition}

We will work with a special type of topological spaces called manfiolds.

\begin{definition}[Manifold  \cite{tu2011}]
	A topological space $M$ is \EMPH{locally Euclidean of dimension $n$}
	if every point $p$ in $M$ has a neighborhood $U$ such that there is  a
	homeomorphism  $\phi$ from $U$ into and open  subset of $\R^n$.
	We call the pair $(U,\phi: U\to \R^n)$ a \EMPH{chart}, $U$ a \EMPH{coordinate neighborhood}
	and  $\phi$ a \EMPH{coordinate map}. 
A \EMPH{manifold} is a Hausdorff, second countable, locally Euclidean space.
\end{definition}

We will  consider two and three dimensional manifolds. Two dimensional
manifolds are called \emph{surfaces}.
The symbol $S$ in \eqnref{g-b} is a surface.
In order to perform computations on our manifolds, 
we often want a triangulation of our manifolds.
To define triangulations we need some preliminary definitions.



\begin{definition}[Independent Points]
Let $v_0,v_1,\ldots,v_k$ be points in $\R^n$. We call them \EMPH{affinely dependent}
if there are real numbers $\alpha_0,\alpha_1,\ldots,\alpha_k$, not all 0, such that
$\Sigma_{i=0}^k \alpha_iv_i=0$ and $\Sigma_{i=0}^k \alpha_i=0.$
Otherwise,  $v_0,v_1,\ldots,v_k$ are \EMPH{affinely independent}.

\end{definition}

\begin{definition}[Simplices]
A \EMPH{simplex} $\sigma$ is the convex hull of a finite affinely independent
set $A$ in $\R^n$. The points in  $A$ are  called vertices, the dimension
of  $\sigma$ is $|A|-1$.  The convex hull of a subset of vertices of a simplex
$\sigma$ is a \EMPH{face} of $\sigma$.
\end{definition}

\begin{definition}[Simplicial Complex]
A nonempty family $C$ of simplices is a \EMPH{simplicial complex} if the following
are satisfied:
\begin{itemize}
\item  Each face of any simplex is a simplex.
\item The intersection of $\sigma_1 \cap \sigma_2$ is a face of both $\sigma_1$ and 
$\sigma_2$.
\end{itemize}


\end{definition}

Some simplicial complexes are very similar to others.


\begin{definition}[Homeomorphism]
A  \EMPH{homeomorphism}  of topological spaces $(X_1,\tau_1$ and $(X_2,\tau_2)$
is a bijection $\phi:X_1\to X_2$ such that for every $\phi$ and $\phi^{-1}$ are continuous.
\end{definition}
For two topological spaces $X$ and $Y$ if there exists a  homeomorphism between
$X$ and $Y$ we say $X$ and $Y$ are topologically  equivalent and write  $X\cong Y.$

Manifolds can have a boundary denoted $\partial(M)$.
The dimension of $\partial(M)$ is one less than the dimension of $M$.
We are ready to define a triangulation.

\begin{definition}[Triangulation]
For a topological space $X$ and simplicial complex $C$ if $X\cong C$,
then $C$  is a \EMPH{triangulation} of $X$.
\end{definition}

 A  graph  is \EMPH{planar} if it can be drawn in the plane with intersections only occuring
at vertices.

We include the the following proof from Eppstein's list attributed to Thurston
 \cite{thurston}. For any planar graph we can map the graph on to the two sphere
 using stereographic projection.
 

\begin{theorem}[Euler Characteristic for Planar Graphs]\label{thm:euler}
For any planar graph on the 2-sphere we have $V-E+F=2.$
\end{theorem}

\begin{proof}
If needed, perturb the triangulation so that the north and south poles are 
inside of a two faces and there are no vertical edges. At each vertex place a unit positive
charge, at the center of each edge place a unit negative charge and put a unit positive
charge in the middle of each face. Slam the sphere on the ground so that all charges
on the edges and vertices are moved into the face below them. For faces that do not contain a pole
the net charge will be zero, the northern boundary consists of an alternating sequence
of edges and vertices  beginning  and ending with an edge.
The face containing the north pole has a unit positive charge, and the face containing the south
pole contains positive four units of charge and negative three units of charge.
Thus, the total charge is two.

\end{proof}
