\section{The Gauss-Bonnet Theorem}



The Gauss-Bonnet theorem for regular surfaces states

\begin{theorem}[The Continuous Gauss-Bonnet Theorem] \label{thm:g-b-c}

If $M$ is a regular surface with boundary $\partial M$ then
	$$\int_{M} K dA+ \int_{\partial M} k_g ds + \sum_i \beta_i= 2\pi \chi(M)$$
	where  $K$ is Gaussian curvature,
	 $k_g$ is the geodesic curvature,
	 each $\beta_i$  is an exterior angle at a vertex of the boundary and
	$\chi$ is the Euler characteristic.
\end{theorem}


The  Gauss-Bonnet theorem is  telling us, if we add up curvature
at each vertex the sum will be $2\pi$ times to Euler characteristic.
If we can compute the curvature at every vertex then we can use the theorem
to learn global topological information.
Conversely, if we know the Euler characteristic we can learn about the curvature
at individual points.

\subsection{Two Proofs}
We now include a proof of a combinatorial version of the theorem due to Banchoff
\cite{banchoff_critical_1970}
\todo{Choose a proof}