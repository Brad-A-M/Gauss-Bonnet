\section{Triangulating Noncovex Polytopes}
\label{sec:triangulating}


Often, the mesh representing
an object contains more triangles than is necessary. One then wishes to
discard triangles without altering the geometry of the object \cite{simplify-mesh-1999}.
In three dimensions, meshes consist of tetrahedra. 
In \cite{triangulating-polytope-1990}, Chazell and Palios give an
algorithm to triangulate a nonconvex three-dimensional polytope.
The algorithm runs in $O(n+r^2)$ tetrahedra where $n$ is the number
of vertices and $r$ is the number of reflex edges.

A  \EMPH{polyhedron of genus $g$} is a combinatorial three-manifold with 
boundary homeomorphic to a $g$ holed torus.
For a genus $g$ polyhedron, it is natural to ask if one can 
triangulate the manifold with fewer than $O(n+r^2)$ tetrahedra.


Chazelle and Shouraboura us the 
 Gauss-Bonnet theorem shows that, any polyhedron
can be triangulated with $O(n+r^2)$ tetrahedra, regardless  of 
the genus! Moreover, this bound is tight \cite{tetra-bounds-c-s-1994}.
Here is their application.

Let $P$ be a a simple polytope. An edge $e$ in $P$ is
\EMPH{reflex} the interior angle formed by its two incident faces
is greater than $\pi$.
A vertex is reflex is it is incident to a reflex edge.

\begin{theorem}[Reflex Angles]\label{thm:reflex}

Any polyhedron of genus $g$ must have 
at least $g-1$ reflex dihedral angles. 

\end{theorem}

From \thmref{reflex}, it follows that any polyhedron
with $n$ vertices and $r$ reflex angles
can be triangulated with $O(n+r^2)$ tetrahedra 
and the bound is tight. Notice that this result is independent
of the genus.

Let $T$ be a tetrahedra, here, the curvature at a vertex $k_v$ is defined to
be the sum of the angle defect.
The Euler characteristic of a polyhedron is determined by the genus,
$\chi=2-2g$.
Polyhedra do not have a 1-dimensional boundary, 
so, the Gauss-Bonnet theorem,
$$\sum_vk_v=2\pi (2-2g).$$

We show that
$$g\leq r+1.$$ 
Then, \thmref{reflex} will be a consequence of the following
lemma,

\begin{lemma}\label{lem:reflex-edge}
The number of reflex edges  incident to a vertex $v$  is at least $-k_v.$
\end{lemma}

\begin{proof}
Project the faces incident to $v$ onto a 2-sphere centered at $v$
to obtain a ``polygon" $P$on the sphere made  of great circles.
An edge in $T$ is reflex if and only if it gives  a reflex angle on $P$.

Let $L$ be the length of the curve and $R$ is  number of reflex  angles.
Then

\begin{equation} \label{eqn:length-reflex}
L\leq 2\pi (R+1)
\end{equation}
Note, if $R$ is zero then \eqnref{length-reflex}
tells us that the curvature at a convex point is non-negative.

If $R\neq 0$, we draw arcs of great circles from each reflex point
along the bisector of its reflex angle, and thus decompose the unit sphere
into at most $R+1$ convex regions. (Convex here means
great circles are contained in the region).
Apply the $R=0$ case $R+1$ times.

\end{proof}



