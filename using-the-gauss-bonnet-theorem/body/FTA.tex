\subsection{The Fundamental Theorem of Algebra}
\label{sec:fta}

In this section, we share a proof due to Almira and Romero that
the continuous Gauss-Bonnet theorem implies the
Fundamental Theorem of Algebra \cite{almira_yet_2007}.

The theorem is the following.
\begin{theorem}[Fundamental Theorem of Algebra]\label{thm:fta}
	Let $f:\CC \to \CC$ be a non-constant single-variable polynomial with complex coefficients,
	then $f$ has a least one complex root.
\end{theorem}

We now include some basic ideas and definitions from complex analysis that are used
in the proof. A complex number can be separated into its real and imaginary parts $z=x+iy$
A continuous function $f:\CC\to \CC$ is a map $f(z)=f(x,y)=u(x,y)+iv(x,y)$ where $u$ and $v$ are continuous.
A function $f$ is \EMPH{analytic} at a point $z_0$ if there exists a neighborhood around $z_0$ 
such that $f$ is differentiable at each point in the neighborhood.
The \EMPH{Laplacian} of a function is the sum of the second order partial derivatives, denoted $\Delta f$
or just $\Delta$ if the function is clear,
in our case, $\Delta f=\frac{\partial^2 f}{\partial x^2}+\frac{\partial^2 f}{\partial y^2}$.
If $\Delta f=0$ then the function $f$ is said to be \EMPH{harmonic}.

We prove a lemma
\begin{lemma}[Analytic Implies Real Part is Harmonic]\label{lem:anal-harmonic}
	The real part of an analytic function is harmonic.
\end{lemma}
\begin{proof}
	Analytic complex functions satisfy the Cauchy-Riemann equations meaning
	$$\frac{\partial u}{\partial x}=\frac{\partial v}{\partial y} \textrm{ and } \frac{\partial u}{\partial y}=- \frac{\partial v}		{\partial x}.$$
	Differentiating again and using the fact that the mixed partial derivatives are equal
	we have
	$$\frac{\partial^2 u}{\partial y^2}=-\frac{\partial}{\partial y}\left(\frac{\partial v}{\partial x}\right)
	=-\frac{\partial}{\partial x}\left(\frac{\partial v}{\partial y}\right)=-\frac{\partial^2 u}{\partial x^2}.$$

	And $\Delta \textbf{Re}f=\frac{\partial^2 u}{\partial x^2}+\frac{\partial^2 u}{\partial y^2}=0$.
\end{proof}

We now present the proof of \thmref{fta}.

\begin{proof}
 Let $p(z)=a_0+a_1z+\ldots + a_nz^n$ and, for the sake of contradiction,
 assume that $p(z)\neq 0$ for all $z\in \CC$ and $a_0a_n\neq 0.$
 We can define two new functions using $p(z)$.
 Set $p_*(z)=a_n+a_{n-1}z+\ldots + a_0z^n$ and note that $p_*(z)=z^np(1/z)$ for $z\in \CC\setminus\{0\}$.
 Then let $f(z)=p(z)p_*(z)$ for all $z\in \CC.$
 Then $f$ has the property that
 $$|f\left(\frac{1}{z}\right)|=\frac{1}{|z|^{2n}}|f(z)|,$$ for $z\in \CC\setminus\{0\}.$
 The above property gives a well-defined Riemannian metric $g$ on $\widehat{\CC}$
 where 
 $$g=\frac{1}{|f(z)|^{\frac{2}{n}}}|dz|^2, \textrm{ for } z\in \CC \textrm{ and}$$
 $$g=\frac{1}{|f(1/z)|^{\frac{2}{n}}}|d(1/z)|^2 \textrm{ for } z\in \widehat{\CC}\setminus\{0\}.$$
 
 Now, we have the Gaussian curvature $K$ of $g$ implies that for all $z\in \widehat{\CC}$
 $$\frac{1}{|f(z)|^{\frac{2}{n}}} K =\frac{1}{n}\Delta(\log |f(z)|)=\frac{1}{n}\Delta \textbf{Re} \log(f(z))=0,$$
 for all $z\in \CC.$ 
 
Thus, the Gaussian curvature at every point is zero. However, the Gauss-Bonnet
 theorem states that the sum of the curvature over all $z\in \CC$ is $4\pi$, meaning there is at least
 one point where the curvature must be positive. This is a contradiction.
\end{proof}


