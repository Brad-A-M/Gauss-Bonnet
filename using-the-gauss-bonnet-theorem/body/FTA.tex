\subsection{The Fundamental Theorem of Algebra}
\label{sec:fta}

In this section, we share a proof due to Almira and Romero that
the Gauss-Bonnet theorem implies the
Fundamental Theorem of Algebra \cite{almira_yet_2007}.


The theorem is the following:
\begin{theorem}[Fundamental Theorem of Algebra]\label{thm:fta}
	Let $f:\CC \to \CC$ be a non-constant single-variable polynomial with complex coefficients,
	then $f$ has a least one complex root.
\end{theorem}


We now include ideas and definitions from complex analysis that are used
in the proof. 
A complex number represented $z=x+iy$ where $x$ and $y$ are real
and $i^2=-1.$
A continuous complex function is a map $f:\CC\to \CC$ we often write
 $f(z)=f(x,y)=u(x,y)+iv(x,y)$ where $u$ and $v$ are real valued.
The derivative of a complex function is defined similalry to the derivative of a real valued
function via a limit. The \emph{derivative} of $f$ at $z_0$ is
$$f'(z_0)=\lim_{h\to 0} \frac{f(z_0+h)-f(z_0)}{h}$$
if the limit exists and is undefined if the limit does not exist.
Many of the differentiation rules from calculus apply to complex function,
in particular the sum and power rules apply. Thus, it is straightforward to 
to take derivatives of polynomials.


A function $f$ is \EMPH{analytic} at a point $z_0$ if there exists a neighborhood around $z_0$ 
such that $f$ is differentiable at each point in the neighborhood.
We will see that the fact that $f$ has a derivative in a domain tells
us a lot about the function. For instance, analytic functions satisfy the Cauchy-Riemann 
equations meaning
$$\frac{\partial u}{\partial x}=\frac{\partial v}{\partial y} \textrm{ and } \frac{\partial u}{\partial y}=- \frac{\partial v}{\partial x}.$$
The Cauchy-Riemann equations give a way to test for analyticity, if the equations
are satisfied then the function is analytic.
The equations give a convenient way to compute the derivative, namely
$$f'(z)=\frac{\partial u}{\partial x} +i \frac{\partial v}{\partial x} =\frac{\partial v}{\partial u} -i \frac{\partial u}{\partial y}.$$


The \EMPH{Laplacian} of a function is the sum of the second order partial derivatives, denoted $\Delta f$,
or just $\Delta$ if the function is clear,
in our case, $\Delta f=\frac{\partial^2 f}{\partial x^2}+\frac{\partial^2 f}{\partial y^2}$.
The equations $$\frac{\partial^2 u}{\partial x^2}+\frac{\partial^2 u}{\partial y^2}=
\frac{\partial^2 v}{\partial x^2}+\frac{\partial^2 v}{\partial y^2}=0$$
are call Laplace's equation and it is one of the most famous equations in applied mathematics
\cite{zill_first_2008}.
A function that satisfies Laplace's equation is said to be \EMPH{harmonic}.

We make use of the following lemma
\begin{lemma}[Analytic Implies Real Part is Harmonic]\label{lem:anal-harmonic}
	The real part of an analytic function is harmonic.
\end{lemma}
\begin{proof}
	Since our function is analytic, the Cauchy-Riemann equations are satisfied.
	Differentiating again and using the fact that the mixed partial derivatives are equal,
	we have
	$$\frac{\partial^2 u}{\partial y^2}=-\frac{\partial}{\partial y}\left(\frac{\partial v}{\partial x}\right)
	=-\frac{\partial}{\partial x}\left(\frac{\partial v}{\partial y}\right)=-\frac{\partial^2 u}{\partial x^2}.$$

	And $\Delta \textbf{Re}f=\frac{\partial^2 u}{\partial x^2}+\frac{\partial^2 u}{\partial y^2}=0$.
\end{proof}



Let $f:S_1\to S$ be a differentiable map from a surface to itself.
Such maps do not necessarily preserve geometric measurements (curvature or area), but the change
can be accounted for. We compute geometric quantities by using the tangent plane and
a vector normal to the surface.
Let $S$ be a regular manifold with tangent space at a point $p\in T_pS.$
The local parameterization on charts $(u,v):U\to \R^2$ form a basis for
$T_pS$.

Let $g$ be a smooth map on the charts $(U, (u,v))$ of a regular surface
$g(\frac{\partial}{\partial u},\frac{\partial}{\partial v}):U\to \R.$
A \EMPH{Riemannian metric} assigns to each $p$ a positive-definite inner product
$g_p:T_pS\times T_pS \to \R$ and we get a norm on $T_pS$ given by
$|v|_p=\sqrt{g_p(v,v)}$.
Since we have a Riemannian metric we get a map between the two inner products,
$Df_p:T_pS\to T_{f(p)}S$. Moreover, the inner product in the preimage is equal
to $|\det(Df_p)|$ times the inner product in the image.


We express the curvature using the Laplacian.
Recall a a function is conformal if it locally preserves angles.
We can use conformal maps to obtain a useful formula for curvature.
We follow the derivation given in $\S$ 13.1.3 of \cite{dubrovin_modern_1984}.

Let $S$ be a surface in $\R^3$ parameterized by
$$x=x(p,q), \hspace{1cm}  y=y(p,q)  \hspace{1cm} z=z(p,q).$$
Then, using the  first fundamental form, we have a metric on the surface in $\R^3$
$$d\ell^2=E(du)^2+2Fdudv + G(dv)^2$$
with $g=EG-F^2>0.$ 
We omit the proof of the following, there exists local coordinates $u,v$ for the surface with a metric of the form
$$d\ell^2=f(u,v)(du^2+dv^2).$$ 

We can then compute the curvature using this metric


\begin{theorem}[Gaussian Curvature]\label{thm:log-curve}
	Let $u,v$ be conformal coordinates of a surface $S\subset \R^3$ with
	induced metric 
	$$d\ell^2=g(u,v)(du^2+dv^2),$$
	then Gaussian curvature,  is 
		\begin{equation}\label{eqn:log-curve}
			K=\frac{-1}{2g(u,v)}\Delta(\log g(u,v))
		\end{equation}
		where $\Delta=\frac{\partial^2}{\partial u^2}+\frac{\partial^2}{\partial v^2}$
		is the Laplacian.
\end{theorem}
\begin{proof}
	Let the surface be given by local conformal coordinates $u,v$
	with $r=r(u,v)$ and $r=(x,y,z)$ in $\R^3$ coordinates.
	Our metric is given by $d\ell^2=g(u,v)(du^2+dv^2)$
	we have 
	
	\begin{equation}\label{eqn:log-curve-proof-inners}
		\langle r_u,r_u\rangle = \langle r_v,r_v\rangle=g(u,v), \hspace{1cm} \langle r_u,r_v\rangle 0.
	\end{equation}
	Differentiating with respect to $u$ and $v$ we have
	\begin{equation}\label{eqn:log-curve-proof-firsts}
		\frac{ \partial g(u,v)}{2\partial u}=\langle r_{uu}, r_u\rangle =\langle r_{uv}, r_v\rangle,
		\frac{ \partial g(u,v)}{2\partial v}=\langle r_{vv}, r_u\rangle =\langle r_{uv}, r_u\rangle
		\end{equation}
		and
		\begin{equation}\label{eqn:log-curve-proof-firsts-1}
		\langle r_{uu},r_v\rangle + \langle r_u, r_{uv}\rangle=0=\langle r_{uv},r_v\rangle + \langle r_u, r_{vv}\rangle.
		\end{equation}
	
	
	By \eqnref{log-curve-proof-inners} we can now define an orthonormal set of vectors
	\begin{equation}\label{eqn:log-curve-proof-orthog}
		e_1=\frac{r_u}{\sqrt{g(u,v}}, e_2=\frac{r_v}{\sqrt{g(u,v}}, n=[e_1,e_2].
	\end{equation}
	with $e_1$ and $e_2$ tangent to the surface.
	
	Then using the second fundamental form we have
	$$L=\langle r_{uu},n\rangle , M=\langle r_{uv},n\rangle, N=\langle r_{vv},n\rangle.$$
	Relative to $e_1, e_2$ and $n$ we have
	
	\begin{equation}\label{eqn:log-curve-proof-seconds}
		r_{uu}=\left(\frac{1}{2\sqrt{g}}\frac{\partial g}{\partial u},\frac{-1}{2\sqrt{g}}\frac{\partial g}{\partial v},L\right),\\
		r_{uv}=\left(\frac{1}{2\sqrt{g}}\frac{\partial g}{\partial v},\frac{1}{2\sqrt{g}}\frac{\partial g}{\partial u},M\right),\\
		r_{uv}=\left(\frac{-1}{2\sqrt{g}}\frac{\partial g}{\partial u},\frac{1}{2\sqrt{g}}\frac{\partial g}{\partial v},N\right).
	\end{equation}
	Then
	$$\langle r_{uu},r_{vv}\rangle -\langle r_{uv},r_{uv}\rangle=LN-M^2-\frac{1}{2g}\left[\left(\frac{\partial g}{\partial v}\right)^2+
	\left(\frac{\partial g}{\partial u}\right)^2\right]$$
	and by \eqnref{log-curve-proof-firsts-1}, we have
	$$\frac{\partial^2 g}{2\partial u^2}=\langle r_{uuv},r_{v}\rangle+\langle r_{uv},r_{uv}\rangle$$
	$$=\frac{\partial}{\partial v}\langle r_{uu},r_{v}\rangle - \langle r_{uu},r_{vv}\rangle+\langle r_{uv},r_{uv}\rangle$$
	$$=-\frac{1}{2}\frac{\partial^2 g}{\partial v^2}-(LN-M^2)+\frac{1}{2g}\left[\left(\frac{\partial g}{\partial v}\right)^2+
	\left(\frac{\partial g}{\partial u}\right)^2\right].$$
	
	Then using the definition of curvature in terms of the first and second fundamental, \eqnref{curve-dets},
	we have
	$$K=\frac{LN-M^2}{g(u,v)^2}=\frac{-1}{2g(u,v)}\Delta \ln g(u,v)$$
	as desired.
\end{proof}







\begin{lemma}[Curvature and the Laplacian (13.1.3 \cite{dubrovin_modern_1984})]\label{lem:lapa-curve}
	If $u,v$ are conformal\todo{define} coordinates on a surface
	embedded in $\R^3$ with metric $d\ell^2=g(u,v)(du^2+dv^2),$
	then the Gaussian curvature is given by
	$$K=-\frac{1}{2g(u,v)}\Delta \ln g(u,v).$$
\end{lemma}



And now the proof of \thmref{fta}.

\begin{proof}
 Let $p(z)=a_0+a_1z+\ldots + a_nz^n$ and, for the sake of contradiction,
 assume that $p(z)\neq 0$ for all $z\in \CC$ and $a_0a_n\neq 0.$
 We can define two new functions using $p(z)$.
 Set $p_*(z)=a_n+a_{n-1}z+\ldots + a_0z^n$ and note that $p_*(z)=z^np(1/z)$ for $z\in \CC\setminus\{0\}$.
 Then let $f(z)=p(z)p_*(z)$ for all $z\in \CC.$
 Then $f$ has the property that
 $$\bigg | f\left(\frac{1}{z}\right) \bigg |=\frac{1}{|z|^{2n}}|f(z)|,$$ for $z\in \CC\setminus\{0\}.$
 The above property gives a well-defined Riemannian metric $g$ on $\widehat{\CC}$
 where 
 $$g=\frac{1}{|f(z)|^{\frac{2}{n}}}|dz|^2, \textrm{ for } z\in \CC \textrm{ and}$$
 $$g=\frac{1}{|f(1/z)|^{\frac{2}{n}}}|d(1/z)|^2 \textrm{ for } z\in \widehat{\CC}\setminus\{0\}.$$
 
 Now, we have the Gaussian curvature $K$ of $g$ implies that for all $z\in \widehat{\CC}$
 $$\frac{1}{|f(z)|^{\frac{2}{n}}} K =\frac{1}{n}\Delta(\log |f(z)|)=\frac{1}{n}\Delta \textbf{Re} \log(f(z))=0,$$
 for all $z\in \CC.$ 
 
Thus, the Gaussian curvature at every point is zero. However, the Gauss-Bonnet
 theorem states that the sum of the curvature over all $z\in \CC$ is $4\pi$, meaning there is at least
 one point where the curvature must be positive. This is a contradiction.
\end{proof}


