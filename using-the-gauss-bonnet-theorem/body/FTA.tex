\subsection{The Fundamental Theorem of Algebra}
\label{sec:fta}

In this section, we share a proof due to Almira and Romero that
the Gauss-Bonnet theorem implies the
Fundamental Theorem of Algebra \cite{almira_yet_2007}.
The theorem is the following:
\begin{theorem}[Fundamental Theorem of Algebra]\label{thm:fta}
	Let $f:\CC \to \CC$ be a non-constant single-variable polynomial with complex coefficients,
	then $f$ has a least one complex root.
\end{theorem}


We now include standard ideas and definitions from complex analysis that are used
in the proof. 
A complex number represented $z=x+iy$ where $x$ and $y$ are real
and $i^2=-1.$
A continuous complex function is a map $f:\CC\to \CC$ we often write
 $f(z)=f(x,y)=u(x,y)+iv(x,y)$ where $u$ is called the real part and $v$ is the imaginary part.
The derivative of a complex function is defined similalry to the derivative of a real valued
function via a limit. The \emph{derivative} of $f$ at $z_0$ is
$$f'(z_0)=\lim_{h\to 0} \frac{f(z_0+h)-f(z_0)}{h}$$
if the limit exists and is undefined if the limit does not exist.
Many of the differentiation rules from calculus apply to complex function,
in particular the sum and power rules apply. Thus, it is straightforward to 
to take derivatives of polynomials.


A function $f$ is \EMPH{analytic} at a point $z_0$ if there exists a neighborhood around $z_0$ 
such that $f$ is differentiable at each point in the neighborhood.
We will see that the fact that $f$ has a derivative in a domain tells
us a lot about the function. For instance, analytic functions satisfy the Cauchy-Riemann 
equations meaning
$$\frac{\partial u}{\partial x}=\frac{\partial v}{\partial y} \textrm{ and } \frac{\partial u}{\partial y}=- \frac{\partial v}{\partial x}.$$
The Cauchy-Riemann equations give a way to test for analyticity, if the equations
are satisfied then the function is analytic.
The equations give a convenient way to compute the derivative, namely
$$f'(z)=\frac{\partial u}{\partial x} +i \frac{\partial v}{\partial x} =\frac{\partial v}{\partial u} -i \frac{\partial u}{\partial y}.$$


The \EMPH{Laplacian} of a function is the sum of the second order partial derivatives, denoted $\Delta f$,
or just $\Delta$ if the function is clear,
in our case, $\Delta f=\frac{\partial^2 f}{\partial x^2}+\frac{\partial^2 f}{\partial y^2}$.
The equations $$\frac{\partial^2 u}{\partial x^2}+\frac{\partial^2 u}{\partial y^2}=
\frac{\partial^2 v}{\partial x^2}+\frac{\partial^2 v}{\partial y^2}=0$$
are call Laplace's equation and are one of the most famous equations in applied mathematics
\cite{zill_first_2008}.
A function that satisfies Laplace's equation is said to be \EMPH{harmonic}.

We will use of the following lemma
\begin{lemma}[Analytic Implies Real Part Harmonic]\label{lem:anal-harmonic}
	The real part of an analytic function is harmonic.
\end{lemma}
\begin{proof}
	Since our function is analytic, the Cauchy-Riemann equations are satisfied.
	Differentiating again and using the fact that the mixed partial derivatives are equal,
	we have
	$$\frac{\partial^2 u}{\partial y^2}=-\frac{\partial}{\partial y}\left(\frac{\partial v}{\partial x}\right)
	=-\frac{\partial}{\partial x}\left(\frac{\partial v}{\partial y}\right)=-\frac{\partial^2 u}{\partial x^2}.$$
	And $$\frac{\partial^2 u}{\partial x^2}+\frac{\partial^2 u}{\partial y^2}=0.$$
\end{proof}

We will need a fact about the complex logarithm function, to this end the 
\emph{complex exponential function} is defined to be
$$e^z=e^x\cos(y)+ie^x\sin(y).$$
This function has many desirable properties: it is differentiable everywhere,
it is equal to its own derivative, it agrees with the real exponential when the
input is restricted to the reals, and satisfies the same algebraic properties
of the real exponential such as $$e^{z_1+z_2}=e^{z_1}e^{z_2}.$$
When we write the function in polar form we see that
$$|e^z|=e^x\hspace{1cm} \textrm{and} \hspace{1cm} \arg(e^z)=y+2\pi n$$
for $n\in \ZZ.$
The complex exponential differs from the real exponential in that 
it is periodic with a imaginary period of $2\pi i.$ This is significant because
we wish to define the inverse. The infinite horizontal
strip $$-\infty<x<\infty, \hspace{1cm} -\pi< y\leq \pi$$ is the \emph{fundamental region}.

Now to define the complex logarithm. Let $\log_e(x)$ denote the real logarithm.
Suppose $$e^w=z,$$
then $|e^w|=|z|$ and $\arg(e^w)=\arg(z)$. Then $u=\log_e|z|$ and $v=\arg(z)$
and we have $$\ln(z)=\log_e|z|+i\arg(z).$$
Notice that since there are an infinite number of arguments for $z$, the logarithm
gives infinitely many values but there is only one value in the fundamental region.
 And now the fact about the complex
logarithm that we will need in the proof of \thmref{fta}.
\begin{lemma}[Complex Logarithm Fact]\label{lem:log-fact}
	For any complex function $f(z)$ we have
	$$\ln(|f(z)|)=\textbf{Re}\ln(f(z)).$$
\end{lemma}
\begin{proof}
	This fact follows from the definition of the complex logarithm.
	The modulus $|f(z)|$ is a real number and so $\ln(|f(z)|)=\log_e(|f(z)|)$.
	On the other hand, $\ln(f(z))=\log_e(|f(z)|)+i\arg(f(z))$ and so
	$\textbf{Re}\ln(f(z))=\log_e(|f(z)|).$
\end{proof}


The proof of \thmref{fta} requires one additional idea. We now show that if we apply a conformal diffeomorphism
to a surface of constant curvature
we can calculate the curvature of the resulting surface.
This result is often referred to as Liouville's equation \cite{deigard-2020,dubrovin_modern_1984,liouville1838}.




%We compute geometric quantities by using the tangent plane and a vector normal to the surface.
%Let $p$ be a point in a tangent space of $S.$
%The local parameterization on charts $(u,v):U\to \R^2$ form a basis for
%$T_pS$.

%Let $g$ be a smooth map on the charts $(U, (u,v))$ of a regular surface
%$g(\frac{\partial}{\partial u},\frac{\partial}{\partial v}):U\to \R.$
%A \EMPH{Riemannian metric} assigns to each $p$ a positive-definite inner product
%$g_p:T_pS\times T_pS \to \R$ and we get a norm on $T_pS$ given by
%$|v|_p=\sqrt{g_p(v,v)}$.
%Since we have a Riemannian metric we get a map between the two inner products,
%$Df_p:T_pS\to T_{f(p)}S$. Moreover, the inner product in the preimage is equal
%to $|\det(Df_p)|$ times the inner product in the image.

Let $f:S\to S$ be a conformal differentiable map with local coordinates $f(u,v)$.
Such maps do not necessarily preserve geometric measurements (curvature or area), but the change
can be accounted for. 
The first fundamental form tells us how the area changed by $f$
$$d\ell^2=Edu^2+2Fdudv + Gdv^2.$$
Since our map is conformal, we can apply \thmref{first-conformal},
and we know $E=G$ and $F=0$ and 
$$d\ell^2=Gdu^2+0dudv + Gdv^2=G(du^2+dv^2).$$
We call $g(u,v)=G$ the \emph{induced metric} of $f$.


\begin{theorem}[Liouville's Theorem]\label{thm:liouville}
	Let $f(u,v)$ be conformal coordinates of a surface with constant curvature $S\subset \R^3$ with
	induced metric 
	$$d\ell^2=g(u,v)(du^2+dv^2),$$
	then Gaussian curvature of the image of $f$,  is 
		\begin{equation}\label{eqn:log-curve}
			K=\frac{-1}{2g(u,v)}\Delta(\log g(u,v))
		\end{equation}
		where $\Delta=\frac{\partial^2}{\partial u^2}+\frac{\partial^2}{\partial v^2}$
		is the Laplacian.
\end{theorem}
\begin{proof}
	Let $f$ and $g$ be as described above. 
	By \eqref{brioschi}, we can express the curvature as
	
	$$K=\frac{\begin{vmatrix}
\frac{-1}{2}g_{vv}-\frac{1}{2}g_{uu} & \frac{1}{2}g_u & -\frac{1}{2}g_v\\
-\frac{1}{2}g_u & g & 0\\
\frac{1}{2}g_v & 0 & g
\end{vmatrix}-\begin{vmatrix}
0 & \frac{1}{2}g_v & \frac{1}{2}g_u\\
\frac{1}{2}g_v & g & 0\\
\frac{1}{2}g_u & 0 & g
\end{vmatrix}}{g^4}.$$
Expanding by cofactors and simplifying gives

\begin{align}
	K&= \frac{-1}{2g}\left(\frac{gg_{vv}-g_v^2}{g^2}+\frac{gg_{uu}-g_u^2}{g^2}\right) \\
       &=\frac{-1}{2g}\left(\left(\frac{g_u}{g}\right)_u+\left(\frac{g_v}{g}\right)_v\right)\\
       &=\frac{-1}{2g}\left(\log(g)_{uu}+\log(g)_{vv}\right).
\end{align}


\end{proof}



And now the proof of \thmref{fta}.

\begin{proof}
 Let $p(z)=a_0+a_1z+\ldots + a_nz^n$ and, for the sake of contradiction,
 assume that $p(z)\neq 0$ for all $z\in \CC$ and $a_0a_n\neq 0.$
 We can define two new functions using $p(z)$.
 Set $p_*(z)=a_n+a_{n-1}z+\ldots + a_0z^n$ and note that $p_*(z)=z^np(1/z)$ for $z\in \CC\setminus\{0\}$.
 Then let $f(z)=p(z)p_*(z)$ for all $z\in \CC.$
 Then $f$ has the property that
 $$\bigg | f\left(\frac{1}{z}\right) \bigg |=\frac{1}{|z|^{2n}}|f(z)|,$$ for $z\in \CC\setminus\{0\}.$
 The above property gives a well-defined metric $g$ on $\widehat{\CC}$
 where 
 $$g=\frac{1}{|f(z)|^{\frac{2}{n}}}|dz|^2, \textrm{ for } z\in \CC \textrm{ and}$$
 $$g=\frac{1}{|f(1/z)|^{\frac{2}{n}}}|d(1/z)|^2 \textrm{ for } z\in \widehat{\CC}\setminus\{0\}.$$
 
 Now, for every $z\in \widehat{\CC}$ we have 
\begin{equation}\label{eq:fta-proof}
\frac{1}{|f(z)|^{\frac{2}{n}}} K =\frac{1}{n}\Delta(\log |f(z)|)=\frac{1}{n}\Delta \textbf{Re} \log(f(z))=0.
\end{equation}

 The first equality in \eqref{fta-proof} is by \thmref{liouville}, the second equality is by our complex
 logarithm fact, \lemref{log-fact}, and the third equality is because the real part of an analytic function
 is harmonic, \lemref{anal-harmonic}.
Thus, the Gaussian curvature at every point is zero. However, the Gauss-Bonnet
 theorem states that the sum of the curvature over all $z\in \CC$ is $4\pi$, meaning there is at least
 one point where the curvature must be positive. This is a contradiction.
\end{proof}


