\section{Removing Noise From A Scanned Object}
\label{sec:removing}



Meshes that are obtained by scanning real objects contain noise.
Most meshes that are generated by scanning require a complete
remeshing \cite{remeshing-2003}.
As a first step in remeshing, the curvature at each
vertex needs to be estimated.

In \cite{mmsb-2003}, Meyer et al., define the gaussian curvature operator
to estimate the curvature at each vertex. Their operator is 
based on a simple application of the Gauss-Bonnet theorem.
The central idea is to cut a disk around each vertex that does not contain
any other vertices. Then, all Gaussian curvature in the removed
disk is occurs at the vertex of interest.

We associate an area around each vertex$v$. 
For each triangle incident to $v$, if the interior 
angle at $v$ is non-obtuse, mark the circumcenter of the triangle
and if the interior angle is obtuse, make the mid point of the edge
opposite of $v$. See \figref{mixed-area} for an illustration.
Denote the area of this polygon by $A_m.$


\begin{figure}[htb]
\centering
\includegraphics[width=.3\textwidth]{meshes/mixed-area}
\caption{The area $A_m$ associated with a vertex $v$.}
\label{fig:mixed-area}
\end{figure}


Then, since we are considering a closed two-disk $A$ we have $\chi(A)=1$.
Let $F_v$ denote the number of faces incident to $v$, 
then by the Gauss-Bonnet theorem we have

$$\int \int_{A_m}K dA +\sum_i^{F_v} \epsilon_i=2\pi$$
where the sum is over the faces incident to $v$.
The Gaussian curvature operator at a vertex $v$ is defined
to be
$$K(v)=\left( 2\pi -\sum_i^{F_v}\epsilon_i\right)/ A_m.$$

The experiments in \cite{mmsb-2003} found that the average
percent error did not exceed $1.3\%$ when using this operator.

