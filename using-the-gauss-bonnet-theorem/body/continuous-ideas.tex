\section{Continuous Definitions}



For a curve in $\R^3$ how can we quantify the curvature at a point on the curve?
Let $\gamma$ denote a curve in $\R^3$ and let $p$ be a point on $\gamma$.
We traverse $\gamma$ at unit speed, $\gamma(t)=(x(t),y(t),z(t))$.  
Let $\vec{T}$ denote the unit tangent vector of $\gamma$ at $p$. 
The curvature is how much the tangent vector $\vec{T}=(x'(t),y'(t),z'(t))$ is rotating. 
This is exactly the second derivative $\gamma''$. Since we traverse $\gamma$
at unit speed, $\gamma'(t)^2=1,$ and by the chain rule, $\gamma'\cdot \gamma''=0,$
so  the second derivative is orthogonal to $\gamma'$. The norm of the second
derivative is the curvature $k$.
Let $n$ denote a unit vector parallel to $\gamma''$, then $\gamma''=k n$.
By taking the cross product of $N$ and $T$ we obtain a vector $B$.
The vectors $T,N$ and $B$ form the \emph{Fernet frame} of $\gamma$ a $p.$


The curvature $k$ of a curve in $\R^3$ is equivalent to the inverse of the radius
of the circle that best approximates the curve at a point. This circle is called
the osculating circle. 
For example, the unit circle in the $xy$-plane, parameterized by $\gamma(t)=(\cos(t),\sin(t),0)$
we have $\gamma'(t)=(-\sin(t),\cos(t),0)$ and $\gamma''(t)=(-\cos(t),-\sin(t),0)$ and $||\gamma''||=1$
which agrees with the inverse of the radius of the osculating circle.
Another example is a straight line,
the curvature is zero and the radius of the osculating circle is infinite.

We will consider one-dimensional curves that are the boundary of two-dimensional
surfaces $S$ in $\R^3.$ In a surface, a \EMPH{geodesic} is a curve that is a shortest path
between two points in the surface. For example, on $\Sp^2$, the equator is a geodesic
and an inhabitant would view a geodesic as a straight line. 
Then, since $S$ is a manifold, each point has a local chart with a normal vector $N$.
We want to compute the rate of rotation of $\gamma$ about $N$.
We project $\gamma'$ onto the tangent plane to $S$ at $p$.
If $T$ is a unit length tangent vector at p, then the vector $U=N\times T$
is orthogonal to both $N$ and $T$.
The \EMPH{geodesic curvature} $k_g$ at a point $p$ is defined to be the norm of the projection
of $k$ onto the tangent space of $S$ at $p$. The geodesic curvature can be computed
by the formula $k_g(t)=||(\gamma''\cdot U)\cdot U||.$

In two-dimensions, we wish to calculate the rate at which the surface
pulls away from the tangent plane.  There are several equivalent ways 
to calculate curvature of a surface.
We can generalize the concept of an osculating circle to an
osculating sphere or we can compute the rate of change of
a normal vector at a point $p$.

Given a surface, we can define a normal vector on the surface at  a point
by considering a local coordinate chart at $p$ with axis $u$ and $v$.
Once we choose a chart we define a clockwise orientation. If the clockwise
orientation can be consistently extended to the entire surface, we say
the surface is \EMPH{orientable}.

For an orientable surface $S$, the map that  $N:S\to \Sp^2$ that sends each
normal in $S$ to the corresponding point on $\Sp^2$ is
the \EMPH{Gauss map}.
The derivative of the Gauss map, $dN(p)$ quantifies the rate of change of
the normal vector ($dN$ is often called the \emph{Weingarten map} \cite{Crane:2013}).
Thus, $dN_p:T_p(S)\to T_{N(p)}(\Sp^2)$, but since $T_p(S)$ and $T_{N(p)}(\Sp^2)$
are parallel we can define $dN_p$ to be a linear map on $T_p(S)$.









