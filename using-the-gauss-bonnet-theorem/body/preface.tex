\section{Introduction}
\label{sec:intro}

This work contains a collection of applications of the 
Discrete Gauss-Bonnet theorem.
I hope that the number of applications continues to grow,
please share any that you feel
ought to  be included\footnote{\text{bradleymccoy@montana.edu}}.
Here we emphasize applications of the \emph{discrete} Gauss-Bonnet
theorem. 
Several applications of the continuous version of the theorem
are given in \cite{doc76}.
For applications of the Gauss-Bonnet theorem in physics see \cite{tirado-physics-apps}.

This work is inspired by Matou\v{s}ek's book \emph{Using the Borsuk-Ulam Theorem}
\cite{jm08}.
In the book Matou\v{s}ek' states that a theorem is a great theorem if there are
\begin{enumerate}[(1)]
\item several different equivalent versions,
\item many different proofs,
\item a host of extensions and generalizations, and
\item numerous interesting applications.
\end{enumerate}

By this standard, the Gauss-Bonnet theorem is a great theorem.
For (1), will state several different versions of the theorem throughout this paper.
As for (2), several proofs exist.
For (3), on example of a generlization is the celebrated Atiyah–Singer index 
theorem is a celebrated example \cite{atiyah_index_1963}.
This work is dedicated to (4).

This paper is organized as follows:
in \secref{I} we introduce definitions and notation that will be used
throughout the paper. We then state and prove the theorem.
In each subsection of \secref{II}, we present an application of the theorem.
In general, the sections containing applications are ordered from simpler to more technical,
and are independent.


