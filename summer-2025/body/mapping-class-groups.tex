\subsection{Mapping Class Groups}
\label{sec:mapping-class}


Let $S_g$ be a closed, connected, orientable surface of genus $g$.
In this section, we consider using the Gauss-Bonnet theorem
to learn about the mapping class group of $S_g$.
Mapping class groups our described in the book by Farb and Margalit \cite{primer}.

The mapping class group of $S_g,$ denoted by
$$\Mod(S_g),$$
is the group of isotopy classes of orientation-preserving homeomorphisms of $S_g,$
that is 
$$\Mod(S_g)=\text{Homeo}^+(S_g)/\text{isotopy}.$$

In $\Mod(S_g)$ two orientation-preserving homeomorphisms are equivalent if
one can be continuously deformed into the other through orientation-preserving
homeomorphisms.

When $g=0, S_g$ is the sphere. Every orientation preserving homeomorphism $\S^2\rightarrow
\S^2$ can be continuously deformed to the identity. Thus, $\Mod(S_0)=\{e\}$ the trivial group.
When $g=1, S_g$ is the torus. Represent the torus as a unit square with opposite sides
identified. Call the bottom of the square essential curve $a$ and the sides essential curve $b.$
Homeomorphisms of the torus send essential simple closed curves to essential closed curves.
So, every homeomorphism sends 
$$a\mapsto pa+qb, b\mapsto ra+sb$$
where $p,q,r,s\in \mathbb{Z}.$ These maps are organized in a matrix
$$\begin{pmatrix} p & a \\ r & x \end{pmatrix},$$
and since area and orientation are preserved, we have the determinant is $1$.
Thus we have
$$\Mod(S_1)\cong \text{SL}(2,\ZZ).$$

For $g\geq 2$ $\Mod(S_g)$ is more difficult to describe. The Gauss-Bonnet
theorem is implies that finite subgroups of have order at most $84(g-1).$

