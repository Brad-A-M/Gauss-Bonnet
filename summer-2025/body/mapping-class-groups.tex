\subsection{Mapping Class Groups}
\label{sec:mapping-class}


Let $S_g$ be a closed, connected, orientable surface of genus $g$.
In this section, we consider using the Gauss-Bonnet theorem
to learn about the mapping class group of $S_g$.
Mapping class groups our described in the book by Farb and Margalit \cite{primer}.

The mapping class group of $S_g,$ denoted by
$$\Mod(S_g),$$
is the group of isotopy classes of orientation-preserving homeomorphisms of $S_g,$
that is 
$$\Mod(S_g)=\text{Homeo}^+(S_g)/\text{isotopy}.$$

In $\Mod(S_g)$ two orientation-preserving homeomorphisms are equivalent if
one can be continuously deformed into the other through orientation-preserving
homeomorphisms.

When $g=0, S_g$ is the sphere. Every orientation preserving homeomorphism $\S^2\rightarrow
\S^2$ can be continuously deformed to the identity. Thus, $\Mod(S_0)=\{e\}$ the trivial group.
When $g=1, S_g$ is the torus. Represent the torus as a unit square with opposite sides
identified. Call the bottom of the square essential curve $a$ and the sides essential curve $b.$
Homeomorphisms of the torus send essential simple closed curves to essential closed curves.
So, every homeomorphism sends 
$$a\mapsto pa+qb, b\mapsto ra+sb$$
where $p,q,r,s\in \mathbb{Z}.$ These maps are organized in a matrix
$$\begin{pmatrix} p & a \\ r & x \end{pmatrix}.$$
The map must send the unit square to another unit square and preserve
the total area of the torus
and since the orientation is also preserved, we have the determinant is $1$.
Thus we have the mapping class group of $S_1$ is a $2\times 2$ matrix
with integer coefficients and determinate one. This group is called
the special linear group denoted
$$\Mod(S_1)\cong \text{SL}(2,\ZZ).$$

For $g\geq 2$ $\Mod(S_g)$ is more difficult to describe. The Gauss-Bonnet
theorem is implies that finite subgroups of have order at most $84(g-1).$
The Uniformization Theorem states that every closed Riemannian surface admits
a metric of constant curvature. In \secref{polygonal}, we showed that $$\chi(S_g)=2-2g.$$
For $g\geq2,$ this is negative. The Gauss-Bonnet theorem implies
$$\int_{S_g}KdA=2\pi \chi(S_g)$$
Thus, there is a metric where $S_g$ has constant curvature equal to negative one.
One way to see this is to consider $S_g$ as a regular $4g$-gon
identifying edges with the genus $g$ pattern. The quotient space is $S_g,$
the hyperbolic metric on $\mathbb{H}^2$ with constant curvature
negative one descends to a metric on $S_g.$


We next discuss bounding the order
of any finite group of hyperbolic isometries of a genus $g\geq 2$ surface. 

\begin{proposition}
Let $X$ be a hyperbolic surface homeomorphic to $S_g$ with $g\geq2$.
Then Isom$(X)$ is finite in any hyperbolic metric.
\end{proposition}
A proof is given in \cite{primer} and is beyond the scope of this book.
Notice that this is not true for $g=1$ the tours!

If $X$ is a closed hyperbolic surface of genus $g\geq 2,$ then we study
$$Y=X\slash \text{Isom}^+(X).$$

One can show that $Y$ is a manifold and is in fact itself a hyperbolic surface.
Since Isom$^+(X)$ is a finite group we have
$$\text{Area}(Y)=\text{Area}(X)\slash |\text{Isom}^+(X)|.$$
The Gauss-Bonnet theorem tells us Area$(X)=2\pi(2g-2)$ when we use the metric
with constant negative curvature. It can be shown that Area$(Y)\geq \pi/21$
and thus
$$|\text{Isom}^+(X)\leq 84(g-1).$$
This is remarkable because the number 84 is mysterious and the bound is sharp!

