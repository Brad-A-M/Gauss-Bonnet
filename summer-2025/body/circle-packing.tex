\subsection{Circle Packing}
\label{sec:circle-packing}

In \cite{oh_criteria_2022}, Oh uses the Gauss-Bonnet theorem to classify
circle packing types of disk triangulation graphs embedded into simply
connected domains in $\CC$.

A disk triangulation graph is a graph $G=(V,E)$ that is planar, all interior faces
are triangles, the boundary of $G$ forms a simple closed polygonal chain, and each
vertex has finite degree (locally finite). The graph $G$ can, and often does, have
an infinite number of faces.

Disk triangulation graphs can be classified into two types, parabolic and hyperbolic.
This classification relates how functions and random walks on the graph behave.
If a random walk on the graph returns to the start with probability one (recurrent),
then the graph is \emph{parabolic}. If a random walk has a positive probability of escaping to
infinity (transient) then the graph is \emph{hyperbolic}.

The Koebe-Andreev-Thurston theorem (the circle packing theorem) states
that any planar graph can be realized by a set of interior-disjoint disks corresponding to vertices,
such that two disks are tangent if and only if the corresponding vertices are connected
to each other. Thus, if we are given an infinite triangulated disk graph
$G$, we know we have $P(G)$ a circle packing embedding of $G$.
Then $P(G)$ is parabolic if the image of the circle packing fills the entire plane
and hyperbolic if the image of the circle packing fills a disk.

Suppose we are presented with a graph $G$ of disk triangulation with an infinite 
number of vertices, our task is to determine if $G$ is parabolic or hyperbolic
We use the Gauss-Bonnet theorem to make this determination \cite{oh_criteria_2022}.

For each $n\in\NN$ let $B_n$ be the combinatorial ball of radius $n$
centered at a fixed vertex in $G.$ Let $k_n$ be the degree excess sequence
defined above, let $a_n=\sum_{j=0}^{n-1}(k_j+6)$ for $n=1,2,\ldots$.
Then $G$ is circle-packing parabolic if 

\begin{equation}\label{eqn:cp-parabolic}
\sum_{n=1}^{\infty}\frac{1}{a_n}=\infty,
\end{equation}

and $G$ is recurrent if

\begin{equation}\label{eqn:recurrent}
\sum_{n=1}^{\infty}\frac{1}{a_n+a_{n+1}}=\infty.
\end{equation}

As a consequence we have if

$$k_n=\sum_{v\in B_n}(\deg v -6)\leq c\ln n$$
for sufficiently large $n$, where $c$ is a fixed positive constant,
then $G$is circle-packing parabolic and recurrent.