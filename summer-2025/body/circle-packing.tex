\subsection{Circle Packing}
\label{sec:circle-packing}


Circle packing considers arrangements of circles on surfaces, such that
no two circles overlap. Dense packings are often desirable \cite{chang_simple_2010}.
In dense circle packings, all circles are touching other circles.
Given a circle packing one can construct the contact graph with a vertex for each
circle and an edge connecting tangent circles.

A disk triangulation graph is a graph $G=(V,E)$ that is planar, all interior faces
are triangles, the boundary of $G$ forms a simple closed polygonal chain, and each
vertex has finite degree (locally finite). 
In \todo{find citation} proved the following remarkable theorem.

\begin{theorem}[Koebe-Andreev-Thurston (KAT)]\label{thm:kat}
Every finite, simple, planar graph can be represented as a circle packing in the plane.
\end{theorem}


Given disk triangulation graph the corresponding circle packing guaranteed
by \thmref{kat} assigns angles around all vertices in $V$.
Thus, for each vertex we can compute its discrete curvature given by
the circle packing,
$$K(v)=2\pi -\sum_{\text{angles at } v}\theta_i.$$
But this is our definition of discrete curvature!
The Gauss-Bonnet theorem holds using the circle packing metric on the vertices
of graphs which are used to model many important relationships.




\todo{gauss-bonnet all up in here\cite{gu_discrete_2013}}

In \cite{oh_criteria_2022}, Oh uses the Gauss-Bonnet theorem to classify
circle packing types of disk triangulation graphs embedded into simply
connected domains in $\CC$.

The graph $G$ can, and often does, have
an infinite number of faces.

Disk triangulation graphs can be classified into two types, parabolic and hyperbolic.
This classification relates how functions and random walks on the graph behave.
If a random walk on the graph returns to the start with probability one (recurrent),
then the graph is \emph{parabolic}. If a random walk has a positive probability of escaping to
infinity (transient) then the graph is \emph{hyperbolic}.
A circle packing $P(G)$ is parabolic if the image of the circle packing fills the entire plane
and hyperbolic if the image of the circle packing fills a disk.

Suppose we are presented with a graph $G$ of disk triangulation with an infinite 
number of vertices, our task is to determine if $G$ is parabolic or hyperbolic
We use the Gauss-Bonnet theorem to make this determination \cite{oh_criteria_2022}.

For each $n\in\NN$ let $B_n$ be the combinatorial ball of radius $n$
centered at a fixed vertex in $G.$ Let $k_n$ be the degree excess sequence
defined above, let $a_n=\sum_{j=0}^{n-1}(k_j+6)$ for $n=1,2,\ldots$.
Then $G$ is circle-packing parabolic if 

\begin{equation}\label{eqn:cp-parabolic}
\sum_{n=1}^{\infty}\frac{1}{a_n}=\infty,
\end{equation}

and $G$ is recurrent if

\begin{equation}\label{eqn:recurrent}
\sum_{n=1}^{\infty}\frac{1}{a_n+a_{n+1}}=\infty.
\end{equation}

As a consequence we have if

$$k_n=\sum_{v\in B_n}(\deg v -6)\leq c\ln n$$
for sufficiently large $n$, where $c$ is a fixed positive constant,
then $G$is circle-packing parabolic and recurrent.