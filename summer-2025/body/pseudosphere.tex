\subsection{Pseudosphere}
\label{sec:pseudosphere}

In this section, we share a proof due to
\cite{pseudo-app} that the Euler characteristic of the pseudosphere
is zero.

Define the tractrix, a bike with front wheel on the $x$-axis going from negative infinity
to positive infinity, the back wheel is moving backwards while the front wheel is on the
negative $x$-axis then the back wheel switches to roll forward when the front wheel is
on the positive $x$-axis. The path of the back wheel is the tractrix.

The pseudosphere is the surface formed by rotating the tractrix around the $x$-axis.
The paper rigoursly proves that the pseudosphere has constant Gaussian curvature of
negative one and area of $4\pi$.

The geodesic curvature along the boundary of the truncated (at $x=0$ and $x=a$ $a$ approches
infinity) which turns out to be $2\pi$.
We then plug into the Gauss-Bonnet to determine that
the Euler characteristic of the top half of the pseudosphere is zero!
This implies that that the half-pseudosphere is homeomorphic to the cylinder.


\todo{isn't this obvious from the fact that we are truncating the pseudosphere?}

