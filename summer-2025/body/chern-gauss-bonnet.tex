\section{Chern-Gauss-Bonnet Theorem}
\label{sec:chern-gauss-bonnet}

In this section, we consider higher dimensional generalizations of the
Gauss-Bonnet Theorem. 

Let's recall our definition of curvature at a point $p$ on a surface via the Gauss map.
We considered a `small' neighborhood around $p$ and computed
the area of the map of this neighborhood on the sphere. 

Let $S$ be an even-dimensional compact orientable hypersurface in 
$\R^{2n+1},$ and let

$$G:S\rightarrow \Sp^{2n}$$
be the Gauss map into the unit $2n$-sphere.
The pull-back $G^*\omega_{2n}$ is a $2n$ form on $S$ because
 the volume form \todo{define/discuss forms} of $\Sp^{2n}$ is $\omega_{2n}.$
 We are able to compute areas on $S$ using the same method as we did 
for $n=1.$

For even-dimensional hypersurfaces, our task is to show that 
$$\int_SG^*\omega_{2n}=2\pi \chi(S).$$

The degree of $G,$ denoted $\deg G$ is the number of pre-images in 
$G^{-1}(p)$ for a generic $p$ in $\Sp^2.$
The $2$-sphere has volume $4\pi,$ we have
$$\frac{1}{2\pi}\int_SKdA=\frac{1}{2\pi}(\deg G) 4\pi = 2(\deg G).$$


\cite{li_gauss-bonnet-chern_2011} Nice summary
\cite{chern_simple_1944} Chern's original paper.