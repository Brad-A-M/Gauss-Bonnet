\section{Chern-Gauss-Bonnet Theorem}
\label{sec:chern-gauss-bonnet}

Generalizing the Gauss-Bonnet theorem to all even dimensions
first posed by Hopf \cite{hopf_uber_1926}.
In 1943 Allendoerfer and Weil gave such ageneralization \cite{},
their proof is technical. In 1944 Chern gave a simple intrinsic proof
that we share \cite{chern_simple_1944}. In this section, we introduce
the necessary machinery to discuss the Gauss-Bonnet theorem in higher
dimensions. We then include a proof of the Chern-Gauss-Bonnet theorem
due to Li \cite{li_gauss-bonnet-chern_2011}. 

For a surface, a neighborhood around each point can be mapped 
 to a subset of $\R^2$ by a  homeomorphism. A $n$-dimensional manifold has
the property that each point has a neighborhood that can be mapped
to a subset of $\R^n$ by a homeomorphism.


How should we measure curvature in dimensions greater than two?
In two dimensions we computed the Gaussian curvature as the determinant of
a matrix, the derivative of the Gauss map. We approximated the Gauss map
locally using two vectors in the tangent plane.
In higher dimensions we have the Reimannian curvature tensor which we now define.
In $d$ dimensions, we need $d$ vectors to approximate our manifold \todo{define?}.



A map 
$$f:V_1\times \ldots \times V_d\rightarrow W$$ where $V_1,\ldots, V_kd,W$ are vector spaces
is \emph{multilinear} if, for each $i$, if all variables but $v_i$ are constant, then $f(v_1,\ldots ,v_i,\ldots
v_d)$ is a linear function of $v_i.$

A \emph{Reimannian curvature tensor } is a multilinear map from a choice of basis vectors for the tangent
space at a point to $\R$. 


Another way to think about curvature is by how much the second derivative fails
to commute. In $\R^n$, Schwarz's Theorem tells us that second derivative commute
$$\frac{\partial^2 f}{\partial x\partial y}=\frac{\partial^2 f}{\partial y\partial x}.$$
But this is not the case on a curved space. Suppose $V$ is a vector field and $X$ and $Y$
are coordinate vector fields, $\partial_x,\partial_y$ are
a coordinate basis if $\nabla_X \nabla_Y V$ means differentiate $V$ along $Y$, then differentiate
the result along $X.$
We ask, what does 
$$\nabla_X\nabla_YV-\nabla_Y\nabla_X V$$
equal?
On the sphere, consider longitude and latitude as our coordinates and consider
a vector at a point on the equator. If we move to the north-pole, rotate our vector 90$^\circ$ east,
travel back to the equator and return along the equator to our original point, our vector will have rotated.

Some ideas of exterior algebra in \cite{Crane:2013}. $k$-dimensional volumes are described by a list of $k$-vectors.
0-forms on a manifold $M$ is a smooth function $f:M\rightarrow \R.$
1-forms assign to each point $p$ a map from an element of the the tangent space to $\R$.
For example say $\omega = ydx+xdy$, then at the point $(2,1)$ we have
$\omega_{(2,1)}=1dx+2dy,$ then $\omega_{(2,1)}$ maps a vector $v=(a,b)$ to
$1a+2b.$

The choice of the vectors is not the important part, so long as they span the same space.
Linear subspaces and $k$-vectors are similar, except $k$-vectors have magnitude and an orientation.
A vector has a direction and magnitude and direction, while a line does not.

2-vectors $(u\land v)$ can be thought of as the parallelogram formed by vectors $v$ and $u$ in $\R^3.$
The magnitude is the area and there are two orientations, the `top' and `bottom' of the parallelogram.
We are able to add two 2-vectors by adding the corresponding vectors and we can scale each coordinate.
Thus, 2-forms are multilinear.
Switching the order of the two vectors changes the orientation.

A basis for 0-forms is simply 1, for 1-forms is $\{dx^1,dx^2,dx^3\}$,
for 2-forms is $\{dx^2\land dx^3, dx^3\land dx^1, dx^1\land dx^2\}.$

The \emph{exterior derivative} is an operation that maps $k$-froms to $k+1$-forms.
A zero from is a smooth function $f$ is $df=\frac{\partial f}{\partial x^1}dx^1+\ldots + \frac{\partial f}{\partial x^n}dx^n.$ Given a 1-form $\omega=fdx+gdy,$ the exterior derivative is
$$d\omega=\left(\frac{\partial g}{\partial x}-\frac{\partial f}{\partial y}\right)dx\land dy.$$
To summerize some properties of $d$ include:
its linear $d(\alpha +\beta)=d\alpha +d\beta,$ and Leibniz rule:
$d(\alpha \land \beta)=d\alpha \beta + (-1)^k \alpha \land d\beta,$ if $\alpha$ is a $k$-form,
and $d^2=0.$

\todo{transition here to technical differential topology}

Given a smooth $n$ dimensional manifold, we are able to determine the topology of $M$ by considering
smooth functions on $M$.
Define the chain complex of differential forms:

$$0\rightarrow \Omega^0(M) \xrightarrow{d} \Omega^1(M) \xrightarrow{d} \Omega^2(M)\xrightarrow{d}\ldots$$
where $\Omega^k(M)$ is the space of smooth $k$-forms, and $d$ is the exterior derivative.
Since $d\circ d=0$, we have im$(d:\Omega^{k-1}\rightarrow \Omega^k)\subseteq \ker(d:\Omega^{k}\rightarrow \Omega^{k+1}).$
A $k$-form $\omega$ is \emph{closed} if $d\omega=0$ and \emph{exact} if there exists
a $k+1$-form $\eta$ with $\omega=d\eta.$
Define

$$H_{dR}^k(M)=\frac{\ker(d:\Omega^{k}\rightarrow \Omega^{k+1})}{\text{im}(d:\Omega^{k-1}\rightarrow \Omega^k)}$$
to be the \emph{$k$th de Rham cohomology group}.

Let us compute $H_{dR}^2(\Sp^1).$ A $1$-form is $\omega=f(\theta)d\theta),$ since $\Sp^1$ is one
dimensional and $d\omega$ is a 2-form, all 1-forms are closed. For a 1-form $\omega$ to be exact
there is a smooth function $f$ on $\Sp^1$ with $\omega=f'(\theta)d\theta$.
Thus, exact 1-forms are those whose integral around the circle is zero, there are many 1-forms
whose integral around the circle are non-zero, say $\omega=d\theta$ where $\int_{\Sp^1}\omega=2\pi.$
A basis for $H_{dR}^1(\Sp^1)$ is given by any 1-form whose integral on $\Sp^1$ is nonzero and
$$H_{dR}^1(\Sp^1)\simeq \R.$$



A \emph{sphere bundle} over a manifold is a map
$$\pi:E\rightarrow M$$
such that $\pi^{-1}(p)\simeq \Sp^n.$ 
If we move from one point to another on $M$, a transition function
maps between the spheres associated by the sphere bundle.
If our bundle admits a Riemannian inner product that is preserved
by all transition functions, then we say the bundle can be reduced to $O(d,\R).$

A \emph{section}
is a map
$$s:M\rightarrow E$$
with $\pi \circ s = id_M.$ In other word, for every $p\in M$,
the map $s$ picks and element of $s(p)\in E$ in the fiber over $b$.
If the section is defined on all of $M$, it is a \emph{global section}.

Let $S$ be an oriented sphere bundle over a differentiable manifold $M$.
In general, $S$ might not have a global section, however, there may be a section over
the complement of an isolated set of points $I$, called the singularities of $s$.

\begin{theorem}[Singularities]
Let $\pi: S\rightarrow M$ be a $(d-1)$-sphere bundle over a closed manifold $M$ of dimension $d$.
Assume that the structure group of $S$ can be reduced to the orthogonal group $O(d,\R)$, then there exists
a smooth map $s:M\rightarrow s$ such that $s\in \Gamma(M\setminus I, S),$ where $I$ is discrete.
\end{theorem}

Chern's proof uses a generalized version of \thmref{poincare-index}, the Poincar\'e-Hopf Index Theorem.
The central ideas is that the theorem localizes the global topology of a manifold using the zeros of a vector field.
Let $f:M\rightarrow N$ be a smooth map between two closed oriented manifold of dimension $d$.
Then $f$ defines a pullback on de Rham cohomology $f^*:H_{dR}^d(N,\R)\rightarrow H_{dR}^d(M,\R)$.
Let $\omega$ be the generator of $H_{dR}^d(N,\R),$ then the \emph{mapping degree}
of $f$ is defined to be $\int_Mf^*\omega.$

\begin{theorem}[Gernalized Poincar\'e-Hopf]\label{thm:gph}
Let the set $I$ be defined as above and suppose it is chosen to be discrete.
If $M$ is a closed oriented manifold and $X$ is a vector field on $M$, then
$$\sum_{x_i\in I} \text{ind}_X(x_i)=\chi(M).$$
\end{theorem}


Some comments from historical development Wu.
Chern's contribution - the unit sphere bundle $\pi: S(M)\rightarrow M$
of a Riemannian manifold $M$ can replace the Gauss map.

If $\{\Omega_j^i\}$ is the locally curvature forms in a hypersurface $M$,
then `a computation shows that'

$$G^*\omega_{2n}=\frac{v(\Sp^{2n})}{2}\Omega,$$
where $v(\Sp^{2n})$ is the volume of the unit $2n$-sphere, and $\Omega$
is the globally defined $2n$-form on $M$,
$$\Omega=\frac{1}{2^{2n}\pi^n n!}\sum \epsilon_{i_1\ldots i_{2n}}\Omega_{i_2}^{i_1}\ldots \Omega_{i_{2n}}^{i_{2n-1}}$$ 
and $\epsilon_{i_1\ldots i_{2n}}$ is the sign of the permutation
of $i_1\ldots i_{2n}$.


The unit sphere bundle of an orientable even-dimensional Riemannian manifold 
$M$, there is a $(2n-1)$-form $\Pi$ on $S(M)$,
so that 
$$\pi^*\Omega = d\Pi.$$
When lifted to the sphere bundle, the Gauss-Bonnet integrand $\Omega$ 
becomes exact.

Let $v$ be a unit vector field on $M$ with exactly one singularity at $x_0\in M.$
Then $v$ is a section of $S(M)$ over $M\setminus \{x_0\}.$
Stokes' theorem tells us
$$\int_M\Omega=\int_{S_{x_0}}v^*\Pi.$$

On each fibre $S_{x_0},$ the form $\Pi$ restricts to the volume element of $S_{x_0}$
times a constant, namely

$$\frac{(n-1)!}{2\pi^n}.$$

The Hopf index theorem shows that
$$\int_{S_{x_)}}v^*\Pi=\chi(M).$$ 

Cher's proof that $\pi^*\Omega=d\Pi$ is important.

Let $F(M)$ be the bundle of orthonormal frames of $M$.
The curvature form is $\Theta_j^i$ and $\Theta$ on $F(M)$
be

$$\Theta=\frac{1}{2^{2n}\pi^n n!}\sum \epsilon_{i_1\ldots i_{2n}}\Theta_{i_2}^{i_1}\ldots \Theta_{i_{2n}}^{i_{2n-1}},$$
then, f $\pi_0:F(M)\rightarrow M$ is the natural projection, we have
$$\pi_0^*\Omega =\Theta.$$

Build two $(2n-1)$ forms on $F(M)$
$$\Phi_k=\sum \epsilon_{i_1\ldots i_{2n}} \Theta_{i_2}^{i_1}\land \ldots\land \Theta_{i_{2k}}^{i_{2k-1}}\land \theta_{i_{2n}}^{i_{2k-1}}\land \ldots \land \theta_{i_{2n}}^{i_{2n-1}},$$
and
$$\Psi_k=(2k+1)\sum \epsilon_{i_1\ldots i_{2n}} \Theta_{i_2}^{i_1}\land \ldots\land \Theta_{i_{2k}}^{i_{2k-1}}\land \Theta_{i_{2n}}^{i_{2k-1}}\land \theta_{i_{2n}}^{i_{2k+2}}\land \ldots \land \theta_{i_{2n}}^{i_{2n-1}}.$$

We have the relation:
$$d\Phi_k=-\Psi_{k-1}+\frac{2n-2k-1}{2(k+1)}\Psi_k,$$
where $k=0,\ldots, n-1,$ and $\Psi_{-1}=0.$
Define a $(2n-1)$-form $\Pi'$ on $F(M)$ by
$$\Pi'=\frac{1}{\pi^n}\sum_{k=0}^{n-1}\frac{1}{1\cdot 3\cdot 5\cdot \cdot \cdot (2n-2k-1)\cdot 2^{n+k}k!}\Psi_k.$$

Then $\Psi_{n-1}=(2^{2n}\pi^n n!)\Theta,$ we get
$$d\Pi'=\frac{1}{2^{2n}\pi^n n!}\Psi_{n-1}=\Theta.$$

So, $\Pi'$ is form $\Pi$ on $S(M)$ and $\Theta$ `descends' to $S(M)$ as $\pi^*\Omega.$
Thus,
$$d\Pi=\pi^*\Omega$$
and this completes the proof.
